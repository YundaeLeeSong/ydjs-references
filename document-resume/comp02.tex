% One-page, ATS-friendly LaTeX resume + cover letter (Automation Genesis — Discrete Engineering Track)
% Replace all bracketed placeholders [like this] with your personal details.
\documentclass[11pt]{article}
\hbadness=10000                       % Suppress Warnings (Horizontal)
\vbadness=10000                       % Suppress Warnings (Vertical)
\usepackage[margin=0.7in]{geometry}
\usepackage{parskip}          % better spacing between paragraphs
\usepackage{enumitem}         % control list spacing
\usepackage[hidelinks]{hyperref}
\usepackage{helvet}

% Compact list spacing
\setlist[itemize]{leftmargin=*,itemsep=2pt,topsep=2pt}
\setlist[enumerate]{leftmargin=*,itemsep=2pt,topsep=2pt}

\begin{document}

% ================= Header =================
\begin{center}
  {\LARGE \textbf{[Your Full Name]}}\\[4pt]
  [City, State] \; $|$ \; (XXX) XXX-XXXX \; $|$ \; \href{mailto:you@example.com}{you@example.com} \; $|$ \; \href{https://linkedin.com/in/yourprofile}{linkedin.com/in/yourprofile} \; $|$ \; \href{https://github.com/yourhandle}{github.com/yourhandle}
\end{center}

% ================= Professional Summary =================
\section*{Professional Summary}
Computer Science B.S. (Graduation: [Month Year]) with extensive software engineering experience across desktop, web, and mobile platforms. Strong full-stack development background (Python, C\#, C/C++, Java, JavaScript/Node), deep Linux customization and tooling experience, and practical familiarity with operating-system concepts and C systems programming. Fast learner for domain-specific tools; eager to apply software and systems skills to industrial automation. Available to start \textbf{June 8, 2026}; willing to relocate and travel as required.

% ================= Education =================
\section*{Education}
\textbf{[University Name]} \hfill [City, State] \\
B.S.\ in Computer Science, \textbf{Graduation: [Month Year]} \\
Relevant coursework: Operating Systems, Networks, Embedded Concepts, Data Structures \& Algorithms, Software Engineering

% ================= Technical Skills =================
\section*{Technical Skills}
\begin{itemize}
  \item \textbf{Languages:} Python, C\#, C/C++, Java, JavaScript (Node, React), SQL, Bash
  \item \textbf{Platforms \& Tools:} Linux (customization, shell, systemd), Docker, Git, CI/CD, Visual Studio, VS Code
  \item \textbf{Application Development:} Desktop (Electron, .NET), Web (React, Node), Mobile (Android, iOS)
  \item \textbf{Systems \& Low-level:} OS concepts, multithreading, memory management, device I/O basics, C systems programming
  \item \textbf{Networking \& Data:} TCP/IP fundamentals, HTTP, WebSockets, basic MQTT/REST experience
  \item \textbf{Testing \& DevOps:} Unit testing, automated builds, telemetry/monitoring, logging pipelines
\end{itemize}

% ================= Experience / Projects =================
\section*{Selected Experience \& Projects}
\textbf{Full-Stack Developer — [Company / Project]} \hfill [Month Year -- Month Year] \\
\begin{itemize}
  \item Built and maintained backend services (Node, Python) and React frontends; implemented REST APIs, authentication, and CI/CD pipelines.
  \item Developed telemetry ingestion and processing pipelines for logs/metrics; delivered dashboards for operational visibility and alerts.
  \item Collaborated with QA and DevOps to automate testing and streamline deployments, reducing release friction.
\end{itemize}

\textbf{Cross-Platform Desktop App — [Project / Company]} \hfill [Month Year -- Month Year] \\
\begin{itemize}
  \item Designed and implemented a cross-platform desktop application (Electron / .NET) integrating with native OS features and external devices.
  \item Improved reliability and update flow; authored user and developer documentation.
\end{itemize}

\textbf{Mobile Application — [Project / Team]} \hfill [Month Year -- Month Year] \\
\begin{itemize}
  \item Implemented core mobile features with offline sync and analytics; optimized performance and battery usage.
  \item Coordinated feature delivery with backend services and conducted user testing.
\end{itemize}

\textbf{Systems / C Programming Project — [Course / Project]} \hfill [Month Year -- Month Year] \\
\begin{itemize}
  \item Implemented C utilities and contributed to Linux tooling customization; worked on multithreaded components and profiled memory/CPU usage.
  \item Wrote unit and integration tests and documented interfaces for reuse.
\end{itemize}

% ================= Leadership & Activities =================
\section*{Leadership \& Activities}
\begin{itemize}
  \item \textbf{Software Club / Hackathons:} Led teams producing prototypes under tight deadlines.
  \item \textbf{Teaching Assistant (Systems / Programming):} Guided labs on C programming and OS concepts.
  \item \textbf{Open Source:} Contributed fixes and documentation to community projects (links on GitHub).
\end{itemize}

% ================= Certifications & Training =================
\section*{Certifications \& Training}
\begin{itemize}
  \item [Optional] Linux / DevOps workshop or certificate (replace with specifics)
  \item Introductory IIoT/OT seminar or relevant industry workshop (if attended)
\end{itemize}

% ================= Additional / Logistics =================
\section*{Additional Information}
\begin{itemize}
  \item \textbf{Availability:} Available to start full-time on \textbf{June 8, 2026}.
  \item \textbf{Relocation:} Willing to relocate for program placements (Mason, OH; Alpharetta, GA; Elk Grove Village, IL; Harleysville, PA). Relocation assistance eligible for moves $>$ 50 miles.
  \item \textbf{Travel:} Comfortable with variable travel (project-dependent).
  \item \textbf{Work Authorization:} Authorized to work in the U.S. without sponsorship. % Replace or remove as appropriate
\end{itemize}

\vspace{6pt}
\noindent\textit{Instructions: Replace all bracketed placeholders (e.g., [Your Full Name], [University Name], dates) with your real details. Keep bullets concise and quantify impact where possible.}

% ================= New page for cover letter =================
\newpage

% ================= Cover Letter =================
\begin{center}
  {\Large \textbf{Cover Letter}}\\[6pt]
  {\normalsize [Your Full Name] \quad $|$ \quad [City, State] \quad $|$ \quad (XXX) XXX-XXXX \quad $|$ \quad \href{mailto:you@example.com}{you@example.com}}
\end{center}

\vspace{10pt}

\noindent [Date: \today]

\vspace{6pt}

\noindent Hiring Manager\\
Siemens Digital Industries — Automation Genesis Program\\

\vspace{8pt}

\noindent Dear Hiring Manager,

\vspace{6pt}

\noindent I am writing to express my interest in the Automation Genesis — Discrete Engineering Track at Siemens Digital Industries. I hold a B.S. in Computer Science (graduation: [Month Year]) and bring extensive hands-on experience building desktop, web, and mobile applications. My background centers on full-stack development, backend systems, telemetry and data pipelines, and Linux customization and tooling. I also have practical systems experience in C programming and operating-system concepts, which gives me a solid foundation for interacting with hardware and low-level interfaces.

\vspace{6pt}

\noindent While my formal background has not focused on PLC programming or IEC 61131-3 ladder logic, I have strong low-level and systems skills (multithreading, memory management, device I/O basics) and a proven ability to learn new engineering tools quickly. I am confident that the structured training in the Automation Genesis program will enable me to become productive with PLCs, VFDs, and motion-control workflows. I am particularly excited by Siemens' emphasis on IIoT and software-driven transformation, where my experience with telemetry pipelines, networking basics (including MQTT/REST), and software integration can provide immediate value.

\vspace{6pt}

\noindent Key strengths I bring:
\begin{itemize}
  \item Rapid software learning curve and broad language/tool fluency (Python, C\#, C/C++, Java, JavaScript).
  \item Systems thinking with Linux customization, C systems programming, and performance profiling experience.
  \item Cross-functional collaboration with QA, DevOps, hardware teams, and product stakeholders.
  \item Commitment to start on \textbf{June 8, 2026}, willingness to relocate, and availability for project travel.
\end{itemize}

\noindent I welcome the opportunity to discuss how my software and systems background can contribute to Siemens' Automation Genesis program. Thank you for considering my application — I look forward to the chance to grow into industrial automation and to contribute to customer-facing engineering solutions.

\vspace{12pt}

\noindent Sincerely,\\[4pt]
\noindent [Your Full Name] \\
\noindent (XXX) XXX-XXXX \; $|$ \; \href{mailto:you@example.com}{you@example.com}



\href{https://www.idea-company.siemens.cloud/businesscard/?id=YW8jUhV8}{Mirkic Siemens Business Card}


































\newpage

\section*{Automation Genesis --- Discrete Engineering Track at Siemens}

The \textbf{``Automation Genesis — Discrete Engineering Track''} is identified as 
an 18-month immersive talent development program (full-time, paid) by Siemens.
\begin{itemize}
  \item \textbf{6-Month Intensive Onboarding:} initial phase, work on projects to build a strong foundation in the Siemens automation portfolio.
  \item \textbf{12-Month Development Phase:} development phase, rotation through different functional areas, work on real customer projects, and the design of technical solutions under the guidance of senior engineers.
\end{itemize}

\section*{Discrete automation by \textbf{Siemens Industry, Inc.}}
\textbf{Discrete automation} in Siemens' Digital Industries Automation division is concerned with the manufacturing of individual, countable
items (e.g., automobiles, electronics, or appliances)

\begin{itemize}
  \item \textbf{Field:} Internal Services / Industrial Automation
  \item \textbf{Pay range:} \$81,000 -- \$84,000 annually (actual offer depends on candidate experience, location, and budget).
  \item \textbf{Benefits:} Health and wellness plans; details referenced at the Siemens benefits page: \url{https://www.benefitsquickstart.com/siemens/index.html}
  \item \textbf{Employment type:} Office / Site only, Full-time, permanent, Entry-level
  \item \textbf{Locations:} Mason, OH (first 6 months); then Alpharetta, GA; Elk Grove Village, IL; and/or Harleysville, PA
  \item Start on \textbf{June 8, 2026}, Mason, OH (first 6 months) 
  \begin{enumerate}[label=\arabic*., leftmargin=*]
    \item 6-month intensive onboarding: instructor-led training + hands-on projects.
    \item 12-month development phase: deepen portfolio expertise, customer engagement, and practical engineering experience.
    \item Transition to a designated full-time role, typically as an Engineer on the Technical Consultant team.
    \item \textbf{Support}: Relocation assistance provided for moves $>$ 50 miles.
    \item \textbf{Travel}: project-dependent; typically 5--10\% in light periods, up to 50\% in busier periods.
  \end{enumerate}
\end{itemize}


\section*{Career Path}
Participants are typically placed into full-time positions such as:
\begin{itemize}
  \item Applications Engineer
  \item Automation Consultant
  \item Controls Engineer
\end{itemize}

\newpage
\subsection*{Core responsibilities}
\begin{itemize}
  \item U.S. work authorization with no sponsorship
  \item willing to commit to required moves.
  \item Strong programming foundation (IEC 61131-3 Ladder Logic, C\#, Python, C/C++, Java, mobile/web, PLCs/HMIs).
  \item (Preferred) Leadership experience (academic or extracurricular).
  \item (Preferred) Strong communication skills and customer focus.
  \item (Preferred) Ethernet networking, statics/dynamics, Windows app development
  \item \textbf{Industrial Networking:} The methods for connecting automation devices (like PLCs, VFDs, and robots) to ensure reliable communication.
  \item \textbf{Functional areas}: software/hardware architecture design, testing, project management, electromechanical Systems
  \item \hrulefill
  \item (Preferred/not qualified) Internship or project experience in industrial/manufacturing environments.
  \item (Preferred/not qualified) Basic understanding of control theory and/or PLCs.
  \item \textbf{Control Theory:} The study of how to manipulate the parameters affecting the behavior of a dynamic system.
  \item (Preferred/not qualified) power electronics, motion control, CNC, robotics.
  \item \textbf{Programmable Logic Controllers (PLCs):} The central control units for automation, used to manage machinery and processes.
  \item \textbf{Variable Frequency Drives (VFDs):} Devices used for controlling the speed of electric motors, which are critical for machine control and energy efficiency.
  \item \textbf{Computer Numerical Control (CNC):} Automation systems specifically for machine tools, such as mills, lathes, and routers.
  \item \textbf{Cybersecurity:} The protection of industrial control systems from cyber threats.
  \item \textbf{Digital IIoT Technology:} The application of the Industrial Internet of Things (IIoT) to gather and analyze data from the factory floor.
  \item Design and implement technical solutions using factory automation, motion control, process automation, and simulation tools.
  \item Consult with customers to drive digital transformation via IIoT, additive manufacturing, digital drive trains, and robotics.
  \item Strengthen customer relationships and contribute to Siemens' growth and reputation.
\end{itemize}

In these roles, individuals are responsible for consulting with Siemens' customers, designing custom software and hardware solutions, and assisting companies with the implementation of smart factory and digital transformation initiatives using Siemens technology.




\end{document}