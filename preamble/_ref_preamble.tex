
% paragraph - indentation
\setlength{\parindent}{0pt}                         % default: 18pt





% paragraph - code
\usepackage{listings, lmodern}                               % \begin{lstlisting}\end{lstlisting}
\lstset{
  language=Python,
  commentstyle=\color{darkgreen},
  keywordstyle=\color{blue},
  stringstyle=\color{darkgreen},
  basicstyle=\ttfamily,
  breaklines=true,
  numberstyle=\tiny\color{gray},
  frame=single,
  literate=*{0}{{{\color{blue}0}}}1
            {1}{{{\color{blue}1}}}1
            {2}{{{\color{blue}2}}}1
            {3}{{{\color{blue}3}}}1
            {4}{{{\color{blue}4}}}1
            {5}{{{\color{blue}5}}}1
            {6}{{{\color{blue}6}}}1
            {7}{{{\color{blue}7}}}1
            {8}{{{\color{blue}8}}}1
            {9}{{{\color{blue}9}}}1
}



% Tabular Data (Tables)
\usepackage{array}
\usepackage{lipsum}
\usepackage{longtable}















%%%%%%%%%%%%%%%%%%%%%%%%%%%%%%%%%%%%%%%%%%%%%%%%%%%%%%%%%%
%%%%%%%%%%%%%%%%%%%%% CUSTOM DOCUMENT %%%%%%%%%%%%%%%%%%%%
%%%%%%%%%%%%5%%%%%%%%%%%%%%%%%%%%%%%%%%%%%%%%%%%%%%%%%%%%%
% title pages
\usepackage{outline} 
\usepackage{pmgraph} 
\usepackage[normalem]{ulem}
\usepackage{verbatim}
\usepackage{pgffor}
\usepackage{xstring}
\usepackage{xparse} % for advanced argument parsing
% Signature Fields for Agreement
\usepackage{textcomp} % for \TextField
\usepackage{ifthen}   % for conditional checks
\newcounter{nameCounter}









% ------------------------------------------------------------------------------
% \signField[<Printed Name>][<Width Factor>]: Generates a signature field
% ------------------------------------------------------------------------------
% DEPENDENCIES:
% - hyperref: Required for \TextField command
% - tabbing: Built-in LaTeX environment
%
% DESCRIPTION:
% This command creates a structured signature field with labeled sections for:
% - Signature
% - Printed Name (customizable label)
% - Date
% - Initials
%
% The width of the entire signature field can be adjusted using the second argument.
% The proportions remain the same while scaling.
%
% USAGE:
%   \signField[<Printed Name>][<Width Factor>]
%   - <Printed Name> (Optional): Defaults to "Printed Name"
%   - <Width Factor>: A decimal value specifying the fraction of \textwidth
%
% EXAMPLES:
%   - \shortSignField           
%   - \signField                
%   - \signField[Full Name]     → Uses "Full Name" for the name label by their usage.
% ------------------------------------------------------------------------------
\newcommand{\signField}[1][Printed Name]{
    \stepcounter{nameCounter}
    \begin{tabbing}
        \hspace{0.3\textwidth} \= \hspace{0.35\textwidth} \= \hspace{0.20\textwidth} \= \hspace{0.15\textwidth} \kill
        \makebox[0.3\textwidth]{} \> \TextField[name=name\arabic{nameCounter},width=0.35\textwidth,bordercolor=0 0 0]{} \> \TextField[name=date\arabic{nameCounter},width=0.20\textwidth,bordercolor=0 0 0]{} \> \makebox[0.15\textwidth]{} \\[-8pt]
        \underline{\hspace{0.3\textwidth}} \> \underline{\hspace{0.35\textwidth}} \> \underline{\hspace{0.20\textwidth}} \> \underline{\hspace{0.15\textwidth}} \\ 
        \textit{Signature} \> #1 \> Date \> \makebox[0.15\textwidth][r]{\textbf{\textit{Initials}}}
    \end{tabbing}
}

% ------------------------------------------------------------------------------
% \shortSignField[<Printed Name>]: Generates a compact signature field
% ------------------------------------------------------------------------------
% DEPENDENCIES:
% - hyperref: Required for \TextField command
% - tabbing: Built-in LaTeX environment
%
% DESCRIPTION:
% Creates a simplified signature field with:
% - Signature
% - Printed Name (customizable)
% - Date
%
% USAGE:
%   \shortSignField[<Printed Name>]
%   - <Printed Name> (Optional): Defaults to "Printed Name"
%
% EXAMPLES:
%   - \shortSignField           → Uses default "Printed Name" label
%   - \shortSignField[Name]     → Uses "Name" for the name label
% ------------------------------------------------------------------------------
\newcommand{\shortSignField}[1][Printed Name]{%
    \stepcounter{nameCounter}

    % Define overall width as a multiple of \textwidth
    \newlength{\overallWidth}
    \setlength{\overallWidth}{0.8\textwidth}  % Slightly reduced total width

    % Calculate proportional column widths
    \newlength{\colOne}
    \newlength{\colTwo}
    \newlength{\colThree}

    \setlength{\colOne}{0.3\overallWidth}
    \setlength{\colTwo}{0.4\overallWidth}
    \setlength{\colThree}{0.3\overallWidth}

    \begin{tabbing}
        \hspace{\colOne} \= \hspace{\colTwo} \= \hspace{\colThree} \kill
        \makebox[\colOne]{} \> 
        \TextField[name=name\arabic{nameCounter},width=\colTwo,bordercolor=0 0 0]{} \> 
        \generatedOn \\[-8pt]
        \underline{\hspace{\colOne}} \> 
        \underline{\hspace{\colTwo}} \> 
        \underline{\hspace{\colThree}} \\ 
        \textit{Signature} \> #1 \> Date
    \end{tabbing}
}

% ------------------------------------------------------------------------------
% \customtitle{<Title>}{<Subtitle>}{<Team>}{<Authors>}{<Course>}{<Institution>}{<Professor>}
% ------------------------------------------------------------------------------
% DEPENDENCIES:
% - graphicx: Required for \includegraphics command
% - hyperref: Required for PDF metadata
%
% DESCRIPTION:
% Creates a professional title page with:
% - Institution logo
% - Title and subtitle
% - Team information
% - Author list
% - Course information
% - Institution name
% - Professor name
%
% USAGE:
%   \customtitle{<Title>}{<Subtitle>}{<Team>}{<Authors>}{<Course>}{<Institution>}{<Professor>}
%
% PARAMETERS:
%   - Title: Main title of the document
%   - Subtitle: Secondary title or project description
%   - Team: Team name or identifier
%   - Authors: List of authors (use \convertarraytostring for multiple authors)
%   - Course: Course name and code
%   - Institution: Institution name
%   - Professor: Professor's name
%
% EXAMPLES:
%   \customtitle
%     {Final Team Project: Process Book}
%     {Music Trends and Insights}
%     {Team: Vizualytics}
%     {Author One, Author Two}
%     {CS-4460-B: Introduction to Information Visualization}
%     {Georgia Institute of Technology}
%     {Professor: John Doe}
% ------------------------------------------------------------------------------
\newcommand{\convertarraytostring}[2]{
  \gdef#1{}
  \foreach \x in {#2} {
    \xdef#1{#1 \x \\}
  }
}
\newcommand{\customtitle}[7]{
    \title{
        % \includegraphics[width=0.60\textwidth]{\BookEmblem}\par\vspace{1cm}
        \includegraphics[width=0.50\textwidth]{\BookEmblem} \\ % fixed Logo settings
        \vspace*{0.5in}
        \textbf{#1} \\ 
        \vspace*{0.5in}
        #2
    }
    \author{
        #3 \\ 
        \vspace*{1pt} \\
        #4 \\
        \vspace*{0.5in} \\
        \textbf{#5} \\ 
        \vspace*{1pt} \\
        #6 \\ 
        \vspace*{1pt} \\
        \textbf{#7} \\ 
    }
    \date{\today} % fixed date settings
    \maketitle
    \newpage
}


% \convertarraytostring
% {\titleauthors}                                                                       % Authors
% {Jaehoon Song (lead), Manya Jain, Devika Papal, Yashman P Singh}
% \customtitle
% {Final Team Project: Process Book}                                                    % Title
% {Music Trends and Insights: Visualization of Genres, Artists, and Listener Behaviors} % Subtitle
% {Team: Vizualytics}                                                                   % Team name (e.g., "Team: Vizualytics")
% {\titleauthors}
% {CS-4460-B: Introduction to Information Visualization}                                % Course name (e.g., "CS-4460-B: Introduction to ...")
% {Georgia Institute of Technology}                                                     % Institution (e.g., "Georgia Institute of Technology")
% {Professor: Yalong Yang}                                                              % Professor name (e.g., "Yalong Yang")





















































%%%%%%%%%%%%%%%%%%%%%%%%%%%%%%%%%%%%%%%%%%%%%%%%%%%%%%%%%
% Mathematics
%%%%%%%%%%%%%%%%%%%%%%%%%%%%%%%%%%%%%%%%%%%%%%%%%%%%%%%%%

\newtheorem{definition}{Definition}


















%%%%%%%%%%%%%%%%%%%%%%%%%%%%%%%%%%%%%%%%%%%%%%%%%%%%%%%%%
% Bibliographies (with Citations)
%%%%%%%%%%%%%%%%%%%%%%%%%%%%%%%%%%%%%%%%%%%%%%%%%%%%%%%%%
\usepackage[backend=biber,style=apa]{biblatex}      % \newpage\printbibliography\end{document}
\addbibresource{../references.bib}                      % \cite{reference1}
% \addbibresource{references.bib}                         % \cite{reference1}
%%%%%%%%%%%%%%%%%%%%%%%%%%%%%%%%%%%%%%%%%%%%%%%%%%%%%%%%%
% DEFAULT
%%%%%%%%%%%%%%%%%%%%%%%%%%%%%%%%%%%%%%%%%%%%%%%%%%%%%%%%%
                                                    % \textbf{boldfaced}
                                                    % \textit{intalicized}
%%%% MATH underbrace
% \[
%     a + b + c = \underbrace{x + y + z
%         }_{\text{
%         Sum of terms
%         }
%         }
% \]

















\usepackage[table]{xcolor}
\usepackage{caption}









%%macros for Alg and Uncertainty
\newcommand\p{\mbox{\bf P}\xspace}
\newcommand\np{\mbox{\bf NP}\xspace}
\newcommand{\Alg}{\text{Alg}}
\newcommand{\Opt}{\text{Opt}}

\newcommand{\one}{\mathbf{1}\xspace}
\newcommand{\calD}{\mathcal{D}}
\newcommand{\F}{\mathcal{F}}
\newcommand{\calF}{\mathcal{F}}
\newcommand{\M}{\mathcal{M}}
\newcommand{\reals}{\mathbb{R}}
\newcommand{\sse}{\subseteq}
\newcommand{\Clients}{\text{Clients}}
\newcommand{\X}{\mathbf{X}}
\newcommand{\Y}{\mathbf{Y}}
\newcommand{\I}{\mathbb{I}}
\newcommand{\cc}{\mathbf{c}}
\def \integers {\mathbb{Z}}

\newcommand\OPT{\text{OPT}\xspace}


%%Computational problems

\newcommand\sat{\mbox{SAT}\xspace}

\newcommand\numsat{\mbox{$\sharp$ SAT}\xspace}
%% Notation for integers, natural numbers, reals, fractions, sets, cardinalities
%%and so on

\newcommand{\IGNORE}[1]{}
\newcommand\bz{\mbox{\bf Z}}
\newcommand{\R}{\mathbb{R}}
\newcommand{\E}{\mathbb{E}}
\renewcommand{\P}{\mathbb{P}}
\newcommand\nat{\mbox{\bf N}}
\newcommand\rea{\mbox{\bf R}}
\newcommand\B{\{0,1\}}      % boolean alphabet  use in math mode
\newcommand\Bs{\{0,1\}^*}   % B star            use in math mode
\newcommand\true{\mbox{\sc True}}
\newcommand\false{\mbox{\sc False}}
\DeclareRobustCommand{\fracp}[2]{{#1 \overwithdelims()#2}}
\DeclareRobustCommand{\fracb}[2]{{#1 \overwithdelims[]#2}}
\newcommand{\marginlabel}[1]%
{\mbox{}\marginpar{\it{\raggedleft\hspace{0pt}#1}}}
\newcommand\card[1]{\left| #1 \right|} %cardinality of set S; usage \card{S}
\newcommand\set[1]{\left\{#1\right\}} %usage \set{1,2,3,,}
\newcommand\poly{\mbox{poly}}  %usage \poly(n)
\DeclareMathOperator*{\argmin}{arg\,min}
\DeclareMathOperator*{\argmax}{arg\,max}
\DeclareMathOperator*{\sign}{sign}
\DeclareMathOperator*{\rank}{rank}
\newcommand{\floor}[1]{\lfloor\, $#1$\,\rfloor}
\newcommand{\ceil}[1]{\lceil\, $#1$\,\rceil}
\newcommand{\comp}[1]{\overline{#1}}
\newcommand{\defeq}{\overset{\mathrm{def}}{=}}
\newcommand{\bits}{\set{0,1}}



% Ordinals
% \newcommand{\st}{\textsuperscript{st}\ }
% \newcommand{\nd}{\textsuperscript{nd}\ }
% \newcommand{\rd}{\textsuperscript{rd}\ }
\newcommand{\thsup}{\textsuperscript{th}\ }


\usepackage{qtree} % `\Tree` for simple tree graphing!


\newenvironment{algo}{\vspace{0.07in} \noindent \begin{minipage}	{5.5in}%\begin{quote}
\hrule\vspace{0.01in}\hrule \vspace{0.05in}}{\vspace{-0.0in}\hrule\vspace{0.01in}\hrule \vspace{0.07in}%\end{quote} 
\end{minipage}}%


\usepackage{arydshln} % For dashed/dotted lines
\usepackage{cellspace} % Extra padding in table cells
\usepackage{multicol}

\usepackage{xspace}
\usepackage{braket}





\usepackage{algorithm}
\usepackage{algorithmic}
% \newcommand{\algorithmiccomment}[1]{\quad$\triangleright$ #1}
\makeatletter
\renewcommand{\algorithmiccomment}[1]{\hfill$\triangleright$~#1}
\makeatother
