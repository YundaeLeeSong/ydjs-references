
% ------------------------------------------------------------------------------
% Required Packages for preamble/font.tex
% ------------------------------------------------------------------------------
\makeatletter
% Font encoding and families
\@ifpackageloaded{fontenc}{}{\usepackage[T1]{fontenc}}   % better font encoding
\@ifpackageloaded{lmodern}{}{\usepackage{lmodern}}       % Latin Modern — scalable replacement for CM



% Typography helpers
\@ifpackageloaded{lettrine}{}{\usepackage{lettrine}}      % dropped capitals
\@ifpackageloaded{lipsum}{}{\usepackage{lipsum}}          % dummy text (testing)
\@ifpackageloaded{dirtytalk}{}{\usepackage[
    left = “,% 
    right = ”,% 
    leftsub = ‘,% 
    rightsub = ’ %
]{dirtytalk}}                                               % \say{} quotations

\makeatother

% ------------------------------------------------------------------------------
% Documentation and Usage Examples: lipsum and lettrine
% ------------------------------------------------------------------------------
% DESCRIPTION
%   The lipsum package provides dummy text for testing typographic layout.
%   The lettrine package produces dropped capitals at the beginning of a paragraph.
%
% FEATURES
%   - lipsum: generate one or more paragraphs for quick layout checks
%   - lettrine: control dropped capital lines, spacing, and scaling
%
% USAGE: lipsum
%   Basic:    \lipsum                  % generates a default paragraph
%   Paragraph: \lipsum[1]              % first paragraph
%   Range:     \lipsum[2-3]            % paragraphs 2 to 3
%   Words:     \lipsum[1][1-5]         % words 1–5 from paragraph 1
%
% USAGE: lettrine
%   Basic:     \lettrine{T}{his} starts a paragraph with a dropped “T”
%   Lines:     \lettrine[lines=3]{T}{his}    % drop across 3 lines
%   Spacing:   \lettrine[findent=2pt,nindent=0pt]{T}{his}
%   Scaling:   \lettrine[lines=3,loversize=0.15]{T}{his}
%   Slope:     \lettrine[hang=1]{T}{his}     % tighter hang of text under the drop cap
%
% EXAMPLES (copy into your document body to preview)
%   % Example 1: Simple lipsum
%   % \lipsum[1]
%
%   % Example 2: Variable Length lipsum
%   % \lipsum[2-3]
%
%   % Example 3: Drop cap with default settings
%   % \lettrine{L}{orem} ipsum dolor sit amet, consectetur adipiscing elit.\lipsum[2]
%
%   % Example 4: Three-line drop cap with adjusted size
%   % \lettrine[lines=3,loversize=0.2]{D}{olor} sit amet, consectetur adipiscing elit.\lipsum[3]
%
%   % Example 5: Fine-tuned indents for tighter text wrap
%   % \lettrine[lines=3,findent=1.5pt,nindent=0pt]{T}{ypography} testing block.\lipsum[4]
%
% NOTES
%   - Use \par after a lettrine paragraph if spacing looks off in complex layouts.
%   - Pair lettrine with a serif body font for best aesthetics.






% ------------------------------------------------------------------------------
% \inlinequote{<text>}
% ------------------------------------------------------------------------------
% DESCRIPTION:
%   This command formats text as an inline quote using smart quotes.
%   It provides a quick way to add quoted text within a paragraph.
%
% PARAMETERS:
%   <text>    The text to be quoted (required)
%
% DEPENDENCIES:
%   \usepackage{dirtytalk}
%
% USAGE EXAMPLES:
%   \inlinequote{This is a quoted text}
%   The author said \inlinequote{this is important}.
% ------------------------------------------------------------------------------
\newcommand{\inlinequote}[1]{\say{#1}}








% ------------------------------------------------------------------------------
% \blockquote{<text>}
% ------------------------------------------------------------------------------
% DESCRIPTION:
%   This command creates a block quote with proper spacing and formatting.
%   It centers the quoted text and adds vertical spacing above and below.
%
% FEATURES:
%   - Adds 2.5em spacing above and below the quote
%   - Centers the quoted text
%   - Uses smart quotes for proper typography
%
% PARAMETERS:
%   <text>    The text to be quoted (required)
%
% DEPENDENCIES:
%   \usepackage{dirtytalk}
%
% USAGE EXAMPLES:
%   \blockquote{This is a longer quote that deserves to be set apart}
%   \blockquote{Another important quote}
% ------------------------------------------------------------------------------
\newcommand{\blockquote}[1]{%
  \vspace{2.5em}%
  \begin{quote}%
    \say{#1}%
  \end{quote}%
  \vspace{2.5em}%
}


% ------------------------------------------------------------------------------
% \parenref{<label>}
% ------------------------------------------------------------------------------
% DESCRIPTION:
%   This command produces a reference enclosed in parentheses, e.g., (1.2).
%
% PARAMETERS:
%   <label>    The label to reference (required)
%
% USAGE EXAMPLES:
%   See \parenref{sec:intro} for details.
% ------------------------------------------------------------------------------
\newcommand{\parenref}[1]{(\ref{#1})}


%%%%%%%%%%%%%%%%%%%%%%%%%%%%%%%%%%%%%%%%%%%%%%%%%%%%
% Grouping Shortcuts for VS Code (`View` > `Command Palette` > `Preferences: Open Keyboard Shortcuts (JSON)`)
%%%%%%%%%%%%%%%%%%%%%%%%%%%%%%%%%%%%%%%%%%%%%%%%%%%%
% {
%   "key": "ctrl+b",
%   "command": "editor.action.insertSnippet",
%   "when": "editorLangId == 'latex'",
%   "args": {
%     "snippet": "\\textbf{${TM_SELECTED_TEXT}}"
%   }
% },
% {
%   "key": "ctrl+i",
%   "command": "editor.action.insertSnippet",
%   "when": "editorLangId == 'latex'",
%   "args": {
%     "snippet": "\\textit{${TM_SELECTED_TEXT}}"
%   }
% },
% {
%   "key": "ctrl+u",
%   "command": "editor.action.insertSnippet",
%   "when": "editorLangId == 'latex'",
%   "args": {
%     "snippet": "\\underbar{${TM_SELECTED_TEXT}}"
%   }
% },
% {
%   "key": "ctrl+q",
%   "command": "editor.action.insertSnippet",
%   "when": "editorLangId == 'latex'",
%   "args": {
%     "snippet": "\\inlinequote{${TM_SELECTED_TEXT}}"
%   }
% },
% {
%   "key": "ctrl+shift+q",
%   "command": "editor.action.insertSnippet",
%   "when": "editorLangId == 'latex'",
%   "args": {
%     "snippet": "\\blockquote{${TM_SELECTED_TEXT}}"
%   }
% },
% {
%   "key": "ctrl+shift+f",
%   "command": "editor.action.insertSnippet",
%   "when": "editorLangId == 'latex'",
%   "args": {
%     "snippet": "\\footnote{${TM_SELECTED_TEXT}}"
%   }
% },