
% ------------------------------------------------------------------------------
% Required Packages for preamble/layout-letter.tex
% ------------------------------------------------------------------------------
% Other packages can be conditionally loaded
\makeatletter
\@ifpackageloaded{eso-pic}{}{\usepackage{eso-pic}}
\@ifpackageloaded{xcolor}{}{\usepackage{xcolor}}
\@ifpackageloaded{marginnote}{}{\usepackage{marginnote}}
\@ifpackageloaded{ifoddpage}{}{\usepackage{ifoddpage}}
\@ifpackageloaded{xparse}{}{\usepackage{xparse}}
\makeatother





% geometry should be loaded normally (no makeatletter wrapper needed)
% Load the geometry package
\usepackage[
  % showframe,                        % <--- uncomment for design
  letterpaper
]{geometry}

% ------------------------------------------------------------------------------
% Define commands for the different visual layouts
% ------------------------------------------------------------------------------
% DESCRIPTION
%   These helpers switch between one-sided and two-sided layouts at any point
%   in the document using \newgeometry to apply margins immediately.
%
% COMMANDS
%   \onesidedpaper  – sets equal inner/outer margins (simulates one-sided printing)
%   \twosidedpaper  – enables true two-sided layout with margin notes gutter and a
%                      light vertical guide on the gutter side of each page
%
% USAGE EXAMPLES (copy into your document preamble or body as needed)
%   % Start with one-sided layout
%   % \onesidedpaper
%
%   % Switch to two-sided layout before Chapter 1
%   % \twosidedpaper
%
%   % Show the geometry frame for debugging
%   % Add `showframe` to the geometry options above and recompile
%
% NOTES
%   - \newgeometry takes effect from the next page break. Place the command
%     before content that should use the new layout.
%   - Call \ClearShipoutPictureBG or \AddToShipoutPictureBG only as provided
%     below; they are handled inside the helpers.
%   - When mixing with class options (e.g., `twoside`), prefer these helpers to
%     avoid conflicts.
%
% This command simulates a one-sided layout by setting inner and outer margins to the same value.
\newcommand{\onesidedpaper}{%
  \newgeometry{
    % twoside,
    % includemp,                        % <--- include margin notes in layout calculation
    top=1.0in,
    inner=1.0in,outer=1.0in,
    bottom=1.0in,
  }%
  \ClearShipoutPictureBG % stop showing that material
}

% This command applies a true two-sided layout with different inner and outer margins.
\newcommand{\twosidedpaper}{%
  \newgeometry{
    twoside,
    includemp,                        % <--- include margin notes in layout calculation
    top=1.0in,
    inner=1.0in,marginparsep=0.2in,marginparwidth=1.4in,outer=0.5in,
    bottom=1.0in,
  }%
    \ClearShipoutPictureBG % stop showing that material
    \AddToShipoutPictureBG{% overlay this material under every page
    \AtTextLowerLeft{%
      \begingroup\color{gray!60}%
        \ifodd\value{page}
          \hspace*{\dimexpr\textwidth + 0.5\marginparsep\relax}%
        \else
          \hspace*{\dimexpr-0.5\marginparsep\relax}%
        \fi
        \makebox[0pt][l]{\rule{0.6pt}{1.0\textheight}}%
      \endgroup
    }%
  }%
}




% ------------------------------------------------------------------------------
% Documentation and helper: \sidenote
% ------------------------------------------------------------------------------
% DESCRIPTION
%   Places a short note in the margin on the gutter side (inner margin),
%   adapting automatically to odd/even pages in two-sided layouts. An optional
%   vertical offset allows fine positioning near the reference point.
%
% DEPENDENCIES
%   marginnote, ifoddpage, xparse (for \NewDocumentCommand)
%
% PARAMETERS
%   [<voffset>]   Optional vertical shift (default: 0pt). Positive moves down.
%   {<content>}   The text of the sidenote. Typeset in small, emphasized font.
%
% USAGE EXAMPLES (copy into your document body)
%   % Basic note with no offset
%   % \sidenote{This appears in the inner margin.}
%
%   % Nudge the note down by 6pt
%   % \sidenote[6pt]{Offset note aligned with this paragraph line.}
%
%   % Longer note (multi-line is supported)
%   % \sidenote{A longer sidenote that can span multiple lines and wraps
%   %   within the margin width.}
%
% NOTES
%   - Works best when the document uses a two-sided layout with a clear gutter.
%   - The inner-gutter guide added by \twosidedpaper helps visually align notes.
%   - Avoid very large positive/negative offsets that might collide with header/footer.
\NewDocumentCommand{\sidenote}{O{0pt} m}{%
  % Check if the current page is an odd page
  \checkoddpage
  % Use a conditional statement to place the note on the correct side
  \ifoddpage
    % If it is an odd page (right side), place the note with a right horizontal offset.
    % We use `\marginnote` with its optional arguments.
    % The horizontal offset is applied to the right margin note using `\hspace*`.
    % The vertical offset is applied using the third optional argument of `\marginnote`.
    % The content is wrapped in a `\parbox` to allow for multi-line shifting.
    \marginnote[
      % The optional argument for the left margin note is empty
      ]{%
        % The content for the right margin note
        % Keep box width fixed; shift with zero-width wrapper so positive #2 moves inward
        \makebox[0pt][l]{%
          \hspace*{-0.4\marginparwidth}% odd page: inward = left (offset)
          \parbox[t]{\marginparwidth}{%
            \raggedright%
            \small\emph{#2}%
          }%
        }%
      }[#1]%
  \else
    % even (left): second optional arg = left tweak
    \marginnote[
      % The content for the right margin note
      \small\emph{#2}
      ]{}[#1]%
  \fi
}




%%%%%%%%%%%%%%%%%%%%%%%%%%%%%%%%%%%%%%%%%%%%%%%%%%%%
% Grouping Shortcuts for VS Code (`View` > `Command Palette` > `Preferences: Open Keyboard Shortcuts (JSON)`)
%%%%%%%%%%%%%%%%%%%%%%%%%%%%%%%%%%%%%%%%%%%%%%%%%%%%
% {
%   "key": "ctrl+shift+s",
%   "command": "editor.action.insertSnippet",
%   "when": "editorLangId == 'latex'",
%   "args": {
%     "snippet": "\\sidenote[0pt]{${TM_SELECTED_TEXT}}"
%   }
% },