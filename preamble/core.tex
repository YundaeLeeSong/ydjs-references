%%%%%%%%%%%%%%%%%%%%%%%%%%%%%%%%%%%%%%%%%%%%%%%%%%%%%%%%%%%%%%%%%%%%%%%%%%%%%%%%
% {@requires} VSCode extension: LaTeX Workshop by James Yu
% ------------------------------------------------------------------------------
% `preamble` (core.tex) is a file commonly used in LaTeX projects to define 
% custom configurations and packages for the document. 
%%%%%%%%%%%%%%%%%%%%%%%%%%%%%%%%%%%%%%%%%%%%%%%%%%%%%%%%%%%%%%%%%%%%%%%%%%%%%%%%
%%%%%%%%%%%%%%%%%%%%%%%%%%%%%%%%%%%%%%%%%%%%%%%%%%%%%%%%%%%%%%%%%%%%%%%%%%%%%%%%
%%%%%%%%%%%%%%%%%%%%%%%%%%%%%%%%%%%%%%%%%%%%%%%%%%%%%%%%%%%%%%%%%%%%%%%%%%%%%%%%
% ------------------------------------------------------------------------------
% Required Packages for preamble/core.tex
% ------------------------------------------------------------------------------
\makeatletter
\@ifpackageloaded{graphicx}{}{\usepackage{graphicx}}     % include images
% ======================== Math =======================
% \@ifpackageloaded{amsfonts}{}{\usepackage{amsfonts}}     % math fonts and symbols
% \@ifpackageloaded{amssymb}{}{\usepackage{amssymb}}       % \mathfrak{...}, \square, \blacksquare
% \@ifpackageloaded{amsmath}{}{\usepackage{amsmath}}       % math foundations, align, cases, ...
% \@ifpackageloaded{amsthm}{}{\usepackage{amsthm}}         % theorem/proof environments
\@ifpackageloaded{ifthen}{}{\usepackage{ifthen}}        % basic conditional logic (\ifthenelse)
% ====================== Document =====================
\@ifpackageloaded{fancybox}{}{\usepackage{fancybox}}     % shadowbox, ovalbox, etc.
\@ifpackageloaded{mdframed}{}{\usepackage{mdframed}}     % framed environments
\@ifpackageloaded{datetime2}{}{\usepackage{datetime2}}   % date/time formatting
\@ifpackageloaded{hyperref}{}{\usepackage{hyperref}}     % hyperlinks in PDF
\@ifpackageloaded{enumitem}{}{\usepackage{enumitem}}     % custom enumerate/itemize labels
\@ifpackageloaded{fancyhdr}{}{\usepackage{fancyhdr}}     % custom headers/footers
\@ifpackageloaded{microtype}{}{\usepackage{microtype}}   % improved typography
\@ifpackageloaded{lipsum}{}{\usepackage{lipsum}}         % dummy text for examples
\@ifpackageloaded{xcolor}{}{\usepackage[table]{xcolor}   % colors for text, tables, tikz
  \definecolor{redcolor}{rgb}{1.0, 0.0, 0.0}          
  \definecolor{greencolor}{rgb}{0.3, 1.0, 0.3}        
  \definecolor{bluecolor}{rgb}{0.0, 0.0, 1.0}         
  \definecolor{yellowcolor}{rgb}{1.0, 1.0, 0.0}       
  \definecolor{cyancolor}{rgb}{0.0, 1.0, 1.0}         
  \definecolor{magentacolor}{rgb}{1.0, 0.0, 1.0}      
  \definecolor{blackcolor}{rgb}{0, 0, 0}              
  \definecolor{graycolor}{rgb}{0.2, 0.2, 0.2}         
  \definecolor{darkgreen}{RGB}{0,128,0}               
}
\@ifpackageloaded{changepage}{}{\usepackage{changepage}} % adjust margins (adjustwidth)
\@ifpackageloaded{etoolbox}{}{\usepackage{etoolbox}}     % programming tools for LaTeX
\@ifpackageloaded{xparse}{}{\usepackage{xparse}}         % define flexible macros
% =============== Tabular Data (Tables) ===============
\@ifpackageloaded{tabularx}{}{\usepackage{tabularx}}     % tables with adjustable-width columns
\@ifpackageloaded{array}{}{\usepackage{array}}           % extended array/tabular features
\@ifpackageloaded{longtable}{}{\usepackage{longtable}}   % multipage tables
\@ifpackageloaded{caption}{}{\usepackage{caption}}       % customize captions
\@ifpackageloaded{arydshln}{}{\usepackage{arydshln}}     % dashed/dotted lines in tables
\@ifpackageloaded{multicol}{}{\usepackage{multicol}}     % multiple columns in text
% ================== TikZ Libraries ===================
\@ifpackageloaded{tikz}{}{\usepackage{tikz}}             % diagrams and graphics
\usetikzlibrary{automata, positioning, arrows}
\usetikzlibrary{arrows.meta}
% =====================================================
\makeatother






% ------------------------------------------------------------------------------
% \insertimage{<width>}{<path>}{<caption>}
% ------------------------------------------------------------------------------
% DESCRIPTION:
%   This command inserts an image into the document with automatic formatting,
%   centering, and captioning. The image is sized relative to the text width.
%
% FEATURES:
%   - Automatic figure environment with proper positioning
%   - Centered image with customizable width
%   - Automatic caption and label generation
%   - Responsive sizing using \textwidth
%
% PARAMETERS:
%   {<width>}     Image width as fraction of text width (e.g., 0.8 for 80%)
%   {<path>}      File path relative to current directory (e.g., images/graph.png)
%   {<caption>}   Caption text displayed below the image
%
% DEPENDENCIES:
%   \usepackage{graphicx}
%
% USAGE EXAMPLES:
%   \insertimage{0.8}{images/graph.png}{This is an example caption.}
%   \insertimage{0.5}{logo.png}{Company logo}
%
% RETURNS:
%   Figure reference: \ref{fig:<path>}
% ------------------------------------------------------------------------------
\newcommand{\insertimage}[3]{%
    \begin{figure}[h!]%
        \centering%
        \includegraphics[width=#1\textwidth]{#2}%
        \caption{#3}%
        \label{fig:#2}%
    \end{figure}%
}


% \newcommand{\fullpageimage}[1]{%
%   \begin{adjustwidth}{-\pageMargin}{-\pageMargin}%
%     \includegraphics[width=\pageWidth]{#1}%
%   \end{adjustwidth}%
% }


% ------------------------------------------------------------------------------
% \insertvideo{<width>}{<height>}{<path>}{<caption>}{<autoplay>}{<loop>}
% ------------------------------------------------------------------------------
% DESCRIPTION:
%   This command inserts a video into the document with automatic formatting
%   and media player controls. The video is sized relative to the text width.
%
% FEATURES:
%   - Automatic figure environment with proper positioning
%   - Centered video with customizable dimensions
%   - Click-to-activate media player
%   - Configurable autoplay and loop settings
%   - Automatic caption and label generation
%
% PARAMETERS:
%   {<width>}     Video width as fraction of text width (e.g., 0.8 for 80%)
%   {<height>}    Video height as fraction of text width
%   {<path>}      Video file path (e.g., video.mp4)
%   {<caption>}   Caption text displayed below the video
%   {<autoplay>}  Autoplay setting (true/false)
%   {<loop>}      Loop setting (true/false)
%
% DEPENDENCIES:
%   \usepackage{media9}
%
% USAGE EXAMPLES:
%   \insertvideo{0.8}{0.6}{video.mp4}{Tutorial video}{false}{true}
%   \insertvideo{0.6}{0.4}{demo.mp4}{Product demonstration}{true}{false}
%
% RETURNS:
%   Figure reference: \ref{fig:<path>}
% ------------------------------------------------------------------------------
\newcommand{\insertvideo}[6]{%
    \begin{figure}[h!]%
        \centering%
        \includemedia[%
            width=#1\textwidth,%
            height=#2\textwidth,%
            activate=onclick,%
            addresource=#3,%
            flashvars={%
                src=#3%
                &autoPlay=#5%
                &loop=#6%
            }%
        ]{}{VPlayer.swf}%
        \caption{#4}%
        \label{fig:#3}%
    \end{figure}%
}





% ------------------------------------------------------------------------------
% \generatedOn
% ------------------------------------------------------------------------------
% DESCRIPTION:
%   This command prints the exact date and time when the LaTeX document was 
%   compiled. It uses the `datetime2` package to fetch the timestamp in the ISO 
%   8601 format (YYYY-MM-DD HH:MM:SS).
%
% FEATURES:
%   - Automatically generates current compilation timestamp
%   - Uses ISO 8601 format for consistency
%   - No parameters required
%
% DEPENDENCIES:
%   \usepackage{datetime2}
%
% USAGE EXAMPLES:
%   The document was generated on: \generatedOn
%   Last compiled: \generatedOn
% ------------------------------------------------------------------------------
\newcommand{\generatedOn}{\DTMnow}

% ------------------------------------------------------------------------------
% \handout[<is_shadow>]{<course_name>}{<date>}{<title>}{<author>}{<secondary_info>}{<number>}
% ------------------------------------------------------------------------------
% DESCRIPTION:
%   This command creates a formatted header for handouts with optional shadow effect.
%   It provides a professional-looking title page for course materials.
%
% FEATURES:
%   - Configurable shadow effect (shadowbox vs framebox)
%   - Smart page numbering: only changes if handout number is provided
%   - Responsive width using \textwidth instead of hardcoded values
%   - Professional layout with course information
%   - Index-friendly: preserves page numbering when no handout number given
%   - If <number> is provided: changes page numbering to "number-page" (e.g., "1-1", "1-2")
%   - If <number> is empty: keeps default page numbering (index-friendly)
%
% PARAMETERS:
%   [<is_shadow>]     Boolean for shadow effect (optional, defaults to false)
%   <course_name>     Name of the course (e.g., "CS 4510 Automata and Complexity")
%   <date>            Date of the handout (e.g., "Jan 15, 2025")
%   <title>           Title of the handout (e.g., "Lecture 1")
%   <author>          Name of the author (e.g., "Prof. Alice")
%   <secondary_info>  Additional info (e.g., "Scribe(s): Bob and Charlie")
%   <number>          Handout number or identifier (optional, affects page numbering)
%
% DEPENDENCIES:
%   \usepackage{fancybox}
%   \usepackage{ifthen}
%
% USAGE EXAMPLES:
%   % With handout number (changes page numbering)
%   \newpage\handout
%   {CS 4510}{Jan 15, 2025}
%   {Lecture 1}
%   {Prof. Smith}{Scribe: John}
%   {1}
%
%   % With shadow effect and handout number
%   \newpage\handout[true]
%   {CS 1010}{Feb 1, 2025}
%   {Assignment 1}
%   {Prof. Jones}{Due: Feb 15}
%   {A1}
%
%   % Without handout number (preserves page numbering for index)
%   \newpage\handout
%   {CS 4510}{Jan 15, 2025}
%   {Lecture 1}
%   {Prof. Smith}{Scribe: John}
%   {}
% ------------------------------------------------------------------------------
\newcommand{\handout}[7][false]{%
  % Only change page numbering if handout number (#7) is provided and not empty
  \ifthenelse{\equal{#7}{}}{%
    % No handout number provided - keep default page numbering
  }{%
    % Handout number provided - temporarily change page numbering
    \let\oldthepage\thepage
    \renewcommand{\thepage}{#7-\arabic{page}}%
  }%
  
  \noindent
  \begin{center}%
  \ifthenelse{\equal{#1}{true}}{%
    \makebox[\linewidth]{%
      \shadowbox{%
        \parbox{\dimexpr\linewidth-2\fboxsep-2\fboxrule\relax}{%
          \vbox{%
            \hbox to \linewidth{#2 \hfill #3}%
            \vspace{4mm}%
            \hbox to \linewidth{{\Large \hfill #4 \hfill}}%
            \vspace{2mm}%
            \hbox to \linewidth{{\it #5 \hfill #6}}%
          }%
        }%
      }%
    }%
  }{%
    \makebox[\linewidth]{%
      \framebox[\linewidth]{%
        \parbox{\dimexpr\linewidth-2\fboxsep-2\fboxrule\relax}{%
          \vbox{%
            \hbox to \linewidth{#2 \hfill #3}%
            \vspace{4mm}%
            \hbox to \linewidth{{\Large \hfill #4 \hfill}}%
            \vspace{2mm}%
            \hbox to \linewidth{{\it #5 \hfill #6}}%
          }%
        }%
      }%
    }%
  }%
  \end{center}%
  \vspace*{4mm}%
}



% ------------------------------------------------------------------------------
% \courseheader[<scale>]{<left-top>}{<right-top>}{<title>}{<date>}
% ------------------------------------------------------------------------------
% DESCRIPTION:
%   This command creates a professional course/assignment header using a two-line
%   layout implemented with tabularx. The left column is a fixed-width block
%   controlled by \courseTitleWidth and the right column flexes to fill the line.
%   This implementation ensures strict column alignment and consistent vertical
%   alignment for both rows.
%
% FEATURES:
%   - Column widths controlled explicitly (\courseTitleWidth).
%   - Uses tabularx so rows line up predictably when widths change.
%   - Optional title scaling ('fit') for long/unbreakable text via \resizebox.
%   - Professional typography with ragged text controls to avoid over-centering.
%   - Automatic horizontal rule under the header.
%
% PARAMETERS:
%   [<scale>]      Title scaling option: 'fit' for auto-scaling, omit for natural wrapping
%   {<left-top>}   Small text on the left of the top row (e.g., course name)
%   {<right-top>}  Small text on the right of the top row (e.g., semester)
%   {<title>}      Main title text (centered in left block)
%   {<date>}       Date text (right-aligned on second row)
%
% DEPENDENCIES:
%   \usepackage{tabularx}
%   \usepackage{graphicx} (for scaling functionality)
%   \usepackage{xparse}   (for \NewDocumentCommand)
%   \usepackage{array}    (for column modifiers)
%
% USAGE EXAMPLES:
%   % Set title block width (adjust as needed)
%   \setlength{\courseTitleWidth}{0.70\textwidth}
%
%   % Natural text wrapping (preferred for most cases)
%   \courseheader{Math 2106-B: Foundations of Mathematical Proof}{Fall 2025}%
%                {ICA (In-Class Assignment) 1}{August 18th}
%
%   % Auto-scaling for long titles
%   \courseheader[fit]{Math 2106-B: Foundations of Mathematical Proof}{Fall 2025}%
%                 {Very Long Title That Should Be Scaled}{August 18th}
%
% NOTES:
%   - Adjust \courseTitleWidth to change the left/right split globally.
%   - Use scaling only when natural wrapping is unacceptable.
%   - This tabularx version keeps rows strictly aligned and is a good choice when
%     precise column behavior is desired.
% ------------------------------------------------------------------------------
\newlength{\courseTitleWidth}
\setlength{\courseTitleWidth}{0.70\textwidth}
% Main command (tabularx implementation)
\NewDocumentCommand{\courseheader}{ O{} m m m m }{%
  \noindent
  \begingroup
    \setlength{\parindent}{0em}%
    % tabularx: left column fixed, right column flexible
    \begin{tabularx}{\textwidth}{@{}>{\raggedright\arraybackslash}p{\courseTitleWidth}
                                 @{\extracolsep{\fill}}>{\raggedleft\arraybackslash}X@{}}%
      % Top row (small left / small right)
      {\small #2} & {\small #3} \\[1em]
      % Main row: title (centered in left fixed column) and date (right)
      \multicolumn{1}{@{}p{\courseTitleWidth}@{}}{%
        \centering
        \IfValueTF{#1}{%
          % scaling mode: scale horizontally to the left column width
          \resizebox{\courseTitleWidth}{!}{\Huge\bfseries #4}%
        }{%
          % natural wrapping mode
          {\Huge\bfseries #4}%
        }%
      } &
      {\normalsize #5} \\
    \end{tabularx}\par
    \vspace{3pt}\noindent\rule{\linewidth}{1pt}\vspace{10pt}%
  \endgroup
}





