
%%%%%%%%%%%%%%%%%%%%%%%%%%%%%%%%%%%%%%%%%%%%%%%%%%%%
% Foreign Language Support - Korean
%%%%%%%%%%%%%%%%%%%%%%%%%%%%%%%%%%%%%%%%%%%%%%%%%%%%
% This file provides Korean language support for LaTeX documents.
% It automatically detects the LaTeX engine and loads appropriate packages.
%%%%%%%%%%%%%%%%%%%%%%%%%%%%%%%%%%%%%%%%%%%%%%%%%%%%

% ------------------------------------------------------------------------------
% Required Packages for font-korean.tex
% ------------------------------------------------------------------------------
\makeatletter
% Engine detection and Korean language support
% \@ifpackageloaded{iftex}{}{\usepackage{iftex}}        % Engine detection
% \@ifpackageloaded{fontspec}{}{\usepackage{fontspec}}  % Font specification (XeLaTeX/LuaLaTeX)
% \@ifpackageloaded{xeCJK}{}{\usepackage{xeCJK}}        % CJK support for XeLaTeX/LuaLaTeX
\@ifpackageloaded{inputenc}{}{\usepackage[utf8]{inputenc}}   % characters from various languages and special symbols 
\@ifpackageloaded{kotex}{}{\usepackage{kotex}}        % korean language encoding setup

\@ifpackageloaded{kotex}{}{\usepackage{kotex}}        % korean language encoding setup

\makeatother


% ------------------------------------------------------------------------------
% \kr{<text>}
% ------------------------------------------------------------------------------
% DESCRIPTION:
%   This command provides Korean text support that works across different LaTeX engines.
%   It automatically adapts to the compilation engine being used.
%
% FEATURES:
%   - pdfLaTeX: Uses CJKutf8 package with proper font family
%   - XeLaTeX/LuaLaTeX: Direct text rendering with CJK fonts
%   - Cross-engine compatibility
%
% PARAMETERS:
%   <text>    Korean text to be rendered (required)
%
% DEPENDENCIES:
%   pdfLaTeX: CJKutf8 package
%   XeLaTeX/LuaLaTeX: xeCJK package with CJK fonts
%
% USAGE EXAMPLES:
%   \kr{안녕하세요} - Hello in Korean
%   This is English text with \kr{한국어} mixed in.
%   \kr{수학} means mathematics in Korean.
% ------------------------------------------------------------------------------

\usepackage{iftex}
\ifPDFTeX
  % pdfLaTeX branch (legacy): use CJKutf8
  \usepackage[utf8]{inputenc}
  \usepackage[T1]{fontenc}
  \usepackage{CJKutf8}
  % Usage for pdfLaTeX: \kr{...} wraps Korean text using a CJK family code.
  \newenvironment{foreigntext}[1]{\begin{CJK}{UTF8}{#1}}{\end{CJK}}
  \newcommand{\kr}[1]{\begin{foreigntext}{mj}#1\end{foreigntext}}
\else
  % XeLaTeX / LuaLaTeX branch (recommended)
  \usepackage{fontspec}
  \usepackage{xeCJK}
  % Set fonts (change to fonts installed on your system)
  \setmainfont{TeX Gyre Termes}
  \setsansfont{TeX Gyre Heros}
  \setCJKmainfont{Noto Serif CJK KR}
  % Simple wrapper: under Xe/Lua this is identity
  \newcommand{\kr}[1]{#1}
\fi



%%%%%%%%%%%%%%%%%%%%%%%%%%%%%%%%%%%%%%%%%%%%%%%%%%%%
% Korean Language Shortcuts for VS Code (`View` > `Command Palette` > `Preferences: Open Keyboard Shortcuts (JSON)`)
%%%%%%%%%%%%%%%%%%%%%%%%%%%%%%%%%%%%%%%%%%%%%%%%%%%%
% {
%   "key": "ctrl+shift+k",
%   "command": "editor.action.insertSnippet",
%   "when": "editorLangId == 'latex'",
%   "args": {
%     "snippet": "\\kr{${TM_SELECTED_TEXT}}"
%   }
% },