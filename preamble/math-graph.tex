
% ------------------------------------------------------------------------------
% Required Packages for preamble/math-graph.tex
% ------------------------------------------------------------------------------
\makeatletter
\@ifpackageloaded{xy}{}{\usepackage[all]{xy}}        % XY-pic package for diagrams
\@ifpackageloaded{xparse}{}{\usepackage{xparse}}    % \NewDocumentCommand
\@ifpackageloaded{xstring}{}{\usepackage{xstring}}   % \IfStrEq for string comparison
\@ifpackageloaded{caption}{}{\usepackage{caption}}  % \caption command
\@ifpackageloaded{hyperref}{}{\usepackage{hyperref}} % hyperlinks and cross-references
\makeatother

% ------------------------------------------------------------------------------
% \xyArc[<direction>]
% ------------------------------------------------------------------------------
% DESCRIPTION:
%   Creates a directed arc (arrow) in XY-pic diagrams with specified direction.
%   This is a wrapper around the XY-pic \ar@{->}[<direction>] command.
%
% FEATURES:
%   - Creates directed arrows in XY-pic diagrams
%   - Supports all standard XY-pic direction syntax
%   - Optional direction parameter with default empty
%
% PARAMETERS:
%   [<direction>]    Direction for the arc (optional, e.g., "r", "u", "rr", "ur")
%
% DEPENDENCIES:
%   \usepackage{xy}
%   \usepackage{xparse}
%
% USAGE EXAMPLES:
%   % Basic right arrow
%   A \xyArc[r]
%
%   % Up-right arrow
%   A \xyArc[ur]
% ------------------------------------------------------------------------------
\NewDocumentCommand{\xyArc}{O{}}{\ar@{->}[#1]}

% ------------------------------------------------------------------------------
% \xyEdge[<direction>]
% ------------------------------------------------------------------------------
% DESCRIPTION:
%   Creates an undirected edge (line) in XY-pic diagrams with specified direction.
%   This is a wrapper around the XY-pic \ar@{-}[<direction>] command.
%
% FEATURES:
%   - Creates undirected edges in XY-pic diagrams
%   - Supports all standard XY-pic direction syntax
%   - Optional direction parameter with default empty
%
% PARAMETERS:
%   [<direction>]    Direction for the edge (optional, e.g., "r", "u", "rr", "ur")
%
% DEPENDENCIES:
%   \usepackage{xy}
%   \usepackage{xparse}
%
% USAGE EXAMPLES:
%   % Basic right edge
%   A \xyEdge[r]
%
%   % Up-right edge
%   A \xyEdge[ur]
% ------------------------------------------------------------------------------
\NewDocumentCommand{\xyEdge}{O{}}{\ar@{-}[#1]}

% ------------------------------------------------------------------------------
% \xyLoop[<direction>]
% ------------------------------------------------------------------------------
% DESCRIPTION:
%   Creates a loop (self-loop) in XY-pic diagrams with specified direction.
%   This command provides a convenient way to create loops with various orientations.
%
% FEATURES:
%   - Creates self-loops with configurable direction
%   - Supports 8 cardinal directions plus aliases
%   - Error handling for invalid directions
%   - Default direction is "u" (up)
%
% PARAMETERS:
%   [<direction>]    Loop direction (optional, defaults to "u")
%                    Valid directions: r, ur (ru), u, ul (lu), l, dl (ld), d, dr (rd)
%
% DEPENDENCIES:
%   \usepackage{xy}
%   \usepackage{xparse}
%   \usepackage{xstring}
%
% USAGE EXAMPLES:
%   % Default up loop
%   A \xyLoop
%
%   % Right loop
%   A \xyLoop[r]
%
%   % Up-right loop (two aliases)
%   A \xyLoop[ur]
%   A \xyLoop[ru]
%
%   % Down-left loop (two aliases)
%   A \xyLoop[dl]
%   A \xyLoop[ld]
%
%   % Complex diagram with loops
%   A \xyLoop[r] B \xyLoop[u] C
% ------------------------------------------------------------------------------
\NewDocumentCommand{\xyLoop}{O{u}}{%
  \IfStrEq{#1}{r}{\ar@(dr,ur)[]}{%
  \IfStrEq{#1}{ru}{\ar@(r,ur)[]}{%
  \IfStrEq{#1}{ur}{\ar@(r,ur)[]}{%
  \IfStrEq{#1}{u}{\ar@(ur,ul)[]}{%
  \IfStrEq{#1}{ul}{\ar@(u,ul)[]}{%
  \IfStrEq{#1}{lu}{\ar@(u,ul)[]}{%
  \IfStrEq{#1}{l}{\ar@(ul,dl)[]}{%
  \IfStrEq{#1}{dl}{\ar@(l,dl)[]}{%
  \IfStrEq{#1}{ld}{\ar@(l,dl)[]}{%
  \IfStrEq{#1}{d}{\ar@(dl,dr)[]}{%
  \IfStrEq{#1}{dr}{\ar@(d,dr)[]}{%
  \IfStrEq{#1}{rd}{\ar@(d,dr)[]}{%
    \PackageError{xyLoop}{Invalid loop direction: #1}%
      {Valid directions: r, ur (ru), u, ul (lu), l, dl (ld), d, dr (rd).}%
  }}}}}}}}}}}}
}%





% ------------------------------------------------------------------------------
% COMPREHENSIVE USAGE EXAMPLE WITH CROSS-REFERENCING
% ------------------------------------------------------------------------------
% This section demonstrates how to use all the graph commands together
% and how to create cross-references using \nameref and \pageref.
%
% USAGE EXAMPLES:
% 
%   % Basic graph with nodes and edges
%   \begin{figure}[h]
%     \centering
%     \[
%     \xymatrix@R=1.5em@C=1.5em{
%       A \xyEdge[r]  &   B \xyEdge[r]              & C 
%       \\
%       D \xyEdge[u]  &   E \xyEdge[u]              & F \xyEdge[l]
%       \\
%       G \xyEdge[r]  &   H \xyEdge[r]\xyEdge[ur]   & I
%     }
%     \]
%     \caption{Simple 3x3 grid graph}
%     \label{fig:simple-grid}
%   \end{figure}

%   Here is a reference to the graph: Figure~\ref{fig:mixed-graph}.

%   \begin{figure}[h]
%     \centering
%     \[
%     \xymatrix@R=1.5em@C=1.5em{
%       A \xyArc[r]     & B \xyLoop[u] 
%       \\
%       C \xyEdge[u]    & D \xyArc[ul]      & E \xyArc[l]
%       \\
%       F \xyLoop[d]   & G \xyArc[l]       & H \xyArc[u]
%     }
%     \]
%     \caption{Mixed directed and undirected graph}
%     \label{fig:mixed-graph}
%   \end{figure}

  

%   % Cross-referencing examples
%   As shown in \nameref{fig:simple-grid}, we can create basic grid graphs.
%   The \nameref{fig:mixed-graph} demonstrates both directed and undirected edges.
%   For more complex examples, see \nameref{fig:complex-graph} on page \pageref{fig:complex-graph}.


%   % Complex graph with multiple connections
%   \begin{figure}[h]
%     \centering
%     \[
%     \xymatrix@R=2em@C=2em{
%       % ----------------------------------------------------------------------
%       {} & 
%       *++[o][F]{2} \xyEdge[r]\xyEdge[d] \xyLoop[ul] & 
%       *++[o][F]{3} \xyEdge[l]\xyEdge[ld] \xyLoop[u] & 
%       {} & 
%       *++[o][F]{5} \xyEdge[r] \xyLoop[l] & 
%       *++[o][F]{1} \xyLoop[r] & 
%       {} \\
%       % ----------------------------------------------------------------------
%       {} & 
%       *++[o][F]{6} \xyEdge[u]\xyEdge[ur] \xyLoop[l] & 
%       {} & 
%       *++[o][F]{4} \xyLoop[u] & 
%       {} & 
%       {} & 
%       {} \\
%       % ----------------------------------------------------------------------
%       {} & 
%       {} & 
%       {} & 
%       *++[o][F]{2} \xyEdge[r] \xyEdge[ld] \xyLoop[u] & 
%       *++[o][F]{6} \xyLoop[r] & 
%       {} & 
%       {} \\
%       % ----------------------------------------------------------------------
%       {} & 
%       {} & 
%       *++[o][F]{3} \xyEdge[ur] \xyArc[rr] \xyLoop[l] & 
%       {} & 
%       *++[o][F]{1} \xyEdge[d] \xyLoop[ur] & 
%       {} & 
%       {} \\
%       % ----------------------------------------------------------------------
%       {} & 
%       {} & 
%       {} & 
%       *++[o][F]{4} \xyArc[uur] \xyLoop[ld] \xyLoop[dr] & 
%       *++[o][F]{5} \xyArc[l] \xyLoop[r] & 
%       {} & 
%       {} \\
%     }
%     \]
%     \caption{Complex graph with multiple node types and connections}
%     \label{fig:complex-graph}
%   \end{figure}

%   % Referencing specific graph elements
%   In the \nameref{fig:complex-graph}, node 2 has connections to nodes 3 and 6,
%   while node 4 has multiple loops and arcs to other nodes.
% ------------------------------------------------------------------------------
    