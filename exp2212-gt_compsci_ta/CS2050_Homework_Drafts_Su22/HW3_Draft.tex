\documentclass{article}
\usepackage[utf8]{inputenc}
\usepackage{xcolor}
\usepackage{setspace}
\usepackage{amsmath}
\usepackage{amsfonts}

\title{CS2050 Summer 2022 Homework 3}
\author{Due: June 8}
\date{Released: June 1}

\newcommand{\pt}[1]{\textcolor{blue}{(#1 points)}}

\newcommand{\pte}[1]{\textcolor{blue}{(#1 points each)}}

\newenvironment{solution}
{
\par
\color{blue}
\textbf{Solution:}
}
{
\par
}

\newenvironment{rubric}
{
\par
\begin{spacing}{.6}
\begin{itshape}
\color{red}

}
{
\end{itshape}
\end{spacing}
\par
}

\begin{document}

\maketitle

This assignment is due on \textbf{11:59 PM EST, Wednesday, June 8, 2022}.  On-time submissions receive 3\% extra credit. You may turn it in one day late for 5\% penalty or two days late for a 15\% penalty. Assignments more than two days late will NOT be accepted.  We will prioritize on-time submissions when grading before an exam. \\ 

You should submit a typeset or \emph{neatly} written pdf on Gradescope.  The grading TA should not have to struggle to read what you've written; if your handwriting is hard to decipher, you will be required to typeset your future assignments.\\ 

You may collaborate with other students, but any written work should be your own. Write the names of the students you work with on the top of your assignment.\\

Always justify your work, even if the problem doesn't specify it. It can help the TA's to give you partial credit.



\clearpage
\begin{enumerate}
\item Determine the rule of inference used in each of the following statements. \pte 2
\begin{enumerate}
    \item If you run fast enough, you will start to fly. You did not start to fly. Therefore, you did not run fast enough.
    \item Most reptiles can talk, but only some fish can talk. Therefore, most reptiles can talk.
    \item I will have soup. I will have salad. Therefore, I will have soup and salad.
    \item You are either a freshman, a sophomore, a junior, or a senior. You are not a freshman or a junior or a senior. Therefore, you are a sophomore.
\end{enumerate}

\begin{rubric}
-2, AON
\end{rubric}

\item Using rules of inference, show that the given premises conclude with $d$, citing each step in your argument. \pt{10}

\begin{enumerate}
    \item[1.] $a \lor b$
    \item[2.] $(b \lor c) \rightarrow e$
    \item[3.] $d \lor \neg e$
    \item[4.] $\neg a \lor c$
\end{enumerate}

\begin{rubric}
-2 per incorrect rule invocation
\end{rubric}

\item Using the rules of inference, show that the following premises conclude with “Yoshi wins the race, and Luigi rides a bike.” Be sure to define all propositional variables for full credit (for example, you may define “$t$: Toad gets lost” as one of your propositional variables). Remember, it is possible that you will use all premises, but it is also possible that some are not needed. \pt{12}

\begin{enumerate}
    \item[1.] Toad gets lost, and Luigi rides a bike.
    \item[2.] If Luigi does not ride a bike, then Wario cheats.
    \item[3.] Rosalina is the best princess in the race.
    \item[4.] Wario does not cheat, or Toad does not get lost.
    \item[5.] If Wario does not cheat and Toad gets lost, then Yoshi wins the race.
\end{enumerate}

\begin{rubric}
-4, incorrect predicate declarations

-2, incorrect logical equivalence/citation

-6, doesn't reach conclusion
\end{rubric}

\item The CS 2050 office hours cubicle is moving! The new cubicle has a width of $7x$ and a length of $3y$, where $x, y \in \mathbb{Z}^+$. Prove or disprove that the area of the cubicle is even whenever $x$ is even. Make sure to include the introduction, body, and conclusion of your argument. Clearly state your reasoning for all statements and use a two-column proof for the body whenever possible. \pt{12}

\begin{rubric}
-1, per missing/incorrect citation (cap at -4)

-1, missing domain

-1, uses p,q without defining them

-2, incorrect logic

-2, doesn't use 2-column format

-2, missing conclusion

-4, doesn't reach conclusion
\end{rubric}

\item Use a direct proof to show that if $n + 9$ is odd, then $n^2-3n-8$ is even. Make sure to include the introduction, body, and conclusion. Clearly state your reasoning for all statements and use a two-column proof for the body whenever possible. \pt{12}

\begin{rubric}
-1, per missing/incorrect citation (cap at -4)

-1, missing domain

-1, uses p,q without defining them

-2, incorrect logic

-2, doesn't use 2-column format

-2, missing conclusion

-4, doesn't reach conclusion
\end{rubric}

\item Let $n$ be an integer. Prove the statement “If $7n^2 + 12$ is even, then $n$ is even.” Make sure to include the introduction, body, and conclusion. Clearly state your reasoning for all statements and use a two-column proof for the body whenever possible. 
\begin{enumerate}
    \item Prove the statement using a proof by contrapositive.
    \item Prove the statement using a proof by contradiction.
\end{enumerate}

\begin{rubric}
-1, per missing/incorrect citation (cap at -4)

-1, missing domain

-1, uses p,q without defining them

-2, incorrect logic

-2, doesn't use 2-column format

-2, missing conclusion

-4, doesn't reach conclusion
\end{rubric}

\item Use a proof by cases to show that $\min(a, \min(b,c)) = \min(\min(a,b),c)$, whenever $a,b,c \in \mathbb{R}$. WLOG or similar should NOT appear in your proof. \pte{12}\pt{12}

\begin{rubric}
-1, per missing/incorrect citation (cap at -4)

-1, missing domain

-1, uses p,q without defining them

-2, incorrect logic

-2, doesn't use 2-column format

-2, missing conclusion

-4, doesn't reach conclusion
\end{rubric}

\item Use a proof by contrapositive to show that if $x+y$ is even for $x,y \in \mathbb{Z}$, then $x$ and $y$ have the same parity (i.e. "oddness" or "evenness"). \pt{10}

\begin{rubric}
-1, per missing/incorrect citation (cap at -4)

-1, missing domain

-1, uses p,q without defining them

-2, incorrect logic

-2, doesn't use 2-column format

-2, missing conclusion

-4, doesn't reach conclusion
\end{rubric}

\end{enumerate}

\end{document}