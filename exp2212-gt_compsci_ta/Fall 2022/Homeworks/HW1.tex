\documentclass{article}
\usepackage[utf8]{inputenc}
\usepackage{xcolor}
\usepackage{setspace}

\title{CS2050 Fall 2022 Homework 1}
\author{Due: September 2}
\date{Released: August 26}

\newcommand{\pt}[1]{\textcolor{blue}{(#1 points)}}

\newcommand{\pte}[1]{\textcolor{blue}{(#1 points each)}}

\newenvironment{solution}
{
\par
\color{blue}
\textbf{Solution:}
}
{
\par
}

\newenvironment{rubric}
{
\par
\begin{spacing}{.6}
\begin{itshape}
\color{red}

}
{
\end{itshape}
\end{spacing}
\par
}

\begin{document}

\maketitle

This assignment is due on \textbf{11:59 PM EST, Friday, September 1, 2022}.  On-time submissions receive 2.5 points of extra credit. You may turn it in one day late for a 10 point penalty or two days late for a 25 point penalty. Assignments more than two days late will NOT be accepted.  We will prioritize on-time submissions when grading before an exam. \\ 

You should submit a typeset or \emph{neatly} written pdf on Gradescope.  The grading TA should not have to struggle to read what you've written; if your handwriting is hard to decipher, you will be required to typeset your future assignments.\\ 

You may collaborate with other students, but any written work should be your own. Write the names of the students you work with on the top of your assignment.\\

Always justify your work, even if the problem doesn't specify it. It can help the TA's to give you partial credit.
\\

Author(s): Akshay Kulkarni

\clearpage

\begin{enumerate}

    \item Rewrite each of the following in the form "if ...., then ...".  (You may adjust verb tense as you wish to make the sentences sound natural.) \pt 6
    
    \begin{enumerate}
        \item A sufficient condition for you to pass is that you get an A.
        
        \item You can take the car only if it is charged.
        
        \item A graph must be connected and acyclic for it to be a tree.
    \end{enumerate}
    
    \begin{rubric}
    -2 each, AON (All Or Nothing)
    \end{rubric}
    
    \item Evaluate each of the following propositions as True or False. \pt 8
    \begin{enumerate}
        \item If squares have four sides, then triangles have nine sides.
        \item If $0 \geq 0$, then $0 > -1$.
        \item If $6+2 = 62$, then $2+6=26$.
        \item If $5$ is not odd, then $4$ is not odd. 
    \end{enumerate}
    
    \begin{rubric}
    -2 each, AON
    \end{rubric}
    
    \item Let $s$ be the proposition ``You eat sushi.", let $h$ be the proposition "You hang out with your fiends.", and let $f$ be the proposition "It is Friday." Expressing the following as English sentences. (Again, you may adjust tense as you like.) \pt 6
    
    \begin{enumerate}
        \item $f \rightarrow s$
        \item $\lnot f \wedge (h \vee s)$
    \end{enumerate}
    
    \begin{rubric}
    -3 each, AON
    \end{rubric}
    
    \item Let $l$ be the proposition ``You are late for class.", let $o$ be the proposition ``You oversleep", and let $c$ be the proposition ``You charge your phone." Represent each of the following statements using only $l, o, c$ and logical operators. \pt 6
    
    \begin{enumerate}
        \item You oversleep only when you do not charge your phone.
        \item You are late for class unless you charge your phone and do not oversleep.
    \end{enumerate}
    
    \begin{rubric}
    -3 each, AON
    \end{rubric}
    
    \item Give the converse, contrapositive, and inverse of the statement ``I get ice cream whenever I finish my homework." (Don't worry about tense, just get the idea correct.) \pt 6
    
    \begin{rubric}
    -3 each, AON
    \end{rubric}
    
    \item Construct truth tables for the following propositions. Include all intermediate columns, in an appropriate order, for full credit. \pte 8
    
    \begin{enumerate}
        \item $p \rightarrow \lnot q$
        \item $p \vee (q \wedge \lnot r)$
        
        \item $(p \vee q) \leftrightarrow \lnot (p \wedge q)$
    \end{enumerate}
    
    \begin{rubric}
    -2 per missing/extra/incorrect column
    
    -1 per incorrect cell in a correct column
    
    -2 at least one column to the left of a column it depends on (i.e. $\lnot (p \wedge q)$ to the left of $p \wedge q$).
    
    -1 used 0/1 instead of T/F
    
    Notes: the book gives a truth table for $\leftrightarrow$, so it is okay if they do not make separate columns for the two unidirectional conditionals.
    \end{rubric}
    
    \item Simplify each of the following to $p$, $q$, $T$ or $F$ using logical equivalences. State the equivalence used at each step. \pte 8
    
    \begin{enumerate}
        \item $p \rightarrow (p \vee q)$
        \item $((\lnot p \rightarrow \lnot q) \rightarrow q)$
        \item $(p \rightarrow q) \wedge (p \rightarrow \lnot q)$. (Hint: equivalences can be used in both ``directions".)
    \end{enumerate}
    
    \begin{rubric}
    -1 per incorrect/missing/uncited step
    
    QUESTION: are students required to go from $\lnot p \vee p$ to $p \vee \lnot p$ before negation? What about from $T \vee p$ to $p \vee T$ before domination?
    \end{rubric}
    
    
    
    \item Prove that $p\rightarrow(p\wedge q) \equiv (p \rightarrow q)$ in both of the following ways.
    \begin{enumerate}
        \item truth table \pt{10}
        \item logical equivalences \pt{10}
    \end{enumerate}
    \begin{rubric}
    TRUTH TABLE
    
    -2 per incorrect/missing column
    
    -1 per incorrect cell in a correct column
    
    -2 at least one column to the left of a column it depends on (i.e. $\lnot (p \wedge q)$ to the left of $p \wedge q$).
    
    -1 used 0/1 instead of T/F
    
    -2 no indication of why the table proves the tautology (just circling columns is sufficient).
    
    LOGICAL EQUIVALENCIES
    
    -1 per incorrect/missing/uncited step
    
    -2 no conclusion
    \end{rubric}
    
    
\end{enumerate}

\end{document}