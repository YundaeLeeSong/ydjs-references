\documentclass{article}
\usepackage[utf8]{inputenc}
\usepackage{xcolor}
\usepackage{comment}
\usepackage{setspace}
\usepackage{algpseudocode}
\usepackage{mathtools}

\usepackage{enumitem}

\usepackage{amsmath, amssymb, amsthm}


\title{CS 2050 Fall 2022 Homework 11}
\author{Due: December 2}
\date{Released: November 18}

\newcommand{\pt}[1]{\textcolor{blue}{(#1 points)}}

\newcommand{\pte}[1]{\textcolor{blue}{(#1 points each)}}

\newenvironment{solution}
{
\par
\color{blue}
\textbf{Solution:}
}
{
\par
}

\newenvironment{rubric}
{
\par
\begin{spacing}{.6}
\begin{itshape}
\color{red}

}
{
\end{itshape}
\end{spacing}
\par
}

\begin{document}

\maketitle

\begin{enumerate}
    \item[i.] This assignment is due on \textbf{11:59 PM EST, Friday, December 2, 2022}.  On-time submissions receive 2.5 points of extra credit. You may turn it in one day late for a 10 point penalty or two days late for a 25 point penalty. Assignments more than two days late will NOT be accepted.  We will prioritize on-time submissions when grading before an exam.
    \item[ii.] You will submit your assignment on \textbf{Gradescope} as short answers that will be auto-graded. Please be sure to read the formatting instructions as you submit!
    \item[iii.] You may collaborate with other students, but any written work should be your own. Write the names of the students you work with on the top of your assignment.
\end{enumerate}

Author(s): David Teng, Richard Zhao, Sarthak Mohanty, and Rohan Bodla
\newpage
\begin{enumerate}

\item Assume that there is a party consisting of people between the ages of 18 and 24 inclusive. How many people would have to be at the party to guarantee that at least three people share the same birth date, birth month, and birth year? Note: assume the range of birth years include 2 leap years. \pt{10}
\begin{solution}
There are 7 years, 5 normal years (365 days) and 2 leap years (366 days). In total, this is 2557 days.

You need $2557 \cdot 2 + 1 = 5,115$ days to guarantee 3 people have the same birth date, month, and year.
\end{solution}

\item Assume that a website only allows strings of length 5 as a password for a user account. Valid characters for a password consist of lowercase letters, uppercase letters, digits, or underscores, and the passwords must meet the following requirements:
\begin{itemize}
    \item If the password begins with a vowel, then it must contain exactly 2 distinct digits. Do not consider ``y" a vowel.
    \item If the password begins with a consonant, then the remaining 4 characters must include an even digit.
    \item If the password begins with a number, the product of all numerical digits in the password must be even.
    \item The password cannot begin with an underscore.
\end{itemize} How many unique passwords exist under the given constraints? \pt{8}
\begin{solution}\\
Begins with vowel: $10 \cdot (C^4_2\text{positions for digits}) \cdot$ \\ $(P^{10}_2\text{selections of digits}) \cdot (53^2\text{selections of non-digit characters)} = 15,168,600$

Begins with consonant: $42 \cdot 63^4$\\
Begins with consonant, no even digits: $42 \cdot 58^4$\\
Begins with consonant, at least 1 even digit: $42 \cdot (63^4 - 58^4) = 186,331,530$

For the product of a series of numbers to be odd, they must all be odd\\
Begins with number: $10 \cdot 63^4$\\
Begins with number, no even numbers: $5 \cdot 58^4$\\
Begins with number, at least 1 even number: $10 \cdot 63^4 - 5 \cdot 58^4 = 100,947,130$

Note the constraint that a password cannot begin with an underscore is already satisfied because our three cases exclusively consider passwords that start with vowels, consonants, and numbers. There are no other possibilities for the first character.

Summing up the three cases: $302,447,260$ possible passwords
\end{solution}

\item How many different ways are there to choose 15 donuts from a donut shop that offers them in 21 different flavors?  \pt{5}
\begin{solution}
There are 15 donuts and 21 different flavors. We can think of those boxes as 20 dividers between the boxes, and the donuts between dividers (or between a divider and an edge) are one of the flavors.\\
Now the problem becomes how many ways there are to rearrange 15 donuts and 20 dividers. $C^{35}_{15} = 3,247,943,160$ possible choices
\end{solution}

\item Imagine you are drawing from a deck of 52 cards. Determine the number of ways you can achieve the following 5-card hands drawn from the deck without repeats. \pte{5}
\begin{enumerate}
    \item[a)] A Straight (5 cards of sequential rank). Hint: when considering the Ace, a straight could be Ace, 2, 3, 4, 5 or 10, Jack, Queen, King, Ace, but no other wrap around is allowed (e.g., Queen, King, Ace, 2, 3 is not allowed)
    \item[b)] A Flush (5 cards of the same suit)
    \item[c)] A Full House (3 cards of one rank and 2 from a single other rank)
    \item[d)] A Straight Flush (5 cards of sequential rank from the same suit)
\end{enumerate}
\begin{solution}
\begin{enumerate}
    \item[a)] From Ace-5 to 10-Ace there are 10 different rank combinations for a straight. In each of those straights, each of the cards has one of four suits. There are $10 \cdot 4^5 = 10,240$ possible hands
    \item[b)] There are 4 different suits available for a flush, and in each of those flushes we pick 5 cards from the 13 cards of that suit. There are $4 \cdot C^{13}_5 = 5,148$ possible hands
    \item[c)] There are $P^{13}_2$ ways to choose the rank that will be the 3 of a kind and the rank that will be the pair. Then, there are $C^4_3$ ways to choose the suits of the 3 of a kind and $C^4_2$ ways to choose the suits of the pair. There are $P^{13}_2 \cdot C^4_3 \cdot C^4_2 = 156 \cdot 4 \cdot 6 = 3,744$ possible hands
    \item[d)] From Ace-5 to 10-Ace there are 10 different rank combinations for a straight. All of the 5 cards must be one of four suits. There are $10 \cdot 4 = 40$ possible hands
\end{enumerate}
\end{solution}

\item How many solutions are there to the following equations? \pte{5}
\begin{enumerate}
    \item[a)] $a_1 + a_2 + a_3 + a_4 = 12$\\\\
where $a_1, a_2, a_3,$ and $a_4$ are positive integers\\

    \item[b)] $a_1 + a_2 + a_3 < 13$\\\\
    where $a_1, a_2,$ and $a_3$ are non-negative integers\\

    \item[c)] $a_1 + a_2 + a_3 + a_4 + a_5 \leq 12$\\\\
    where $a_1, a_2, a_3, a_4,$ and $a_5$ are non-negative integers, $a_1 > 5$ and $a_3 \leq 10$\\
    
\end{enumerate}
\begin{solution}
\begin{enumerate}
    \item[a)] Define new variables $b_1 = a_1 - 1, b_2 = a_2 - 1, b_3 = a_3 - 1, b_4 = a_4 - 1$ such that $b_1, b_2, b_3, b_4$ are all non-negative integers. Subtracting 4 from both sides and substituting in $b$ for $a$, the equation becomes:\\
    $b_1 + b_2 + b_3 + b_4 = 8$\\
    There are 8 ones and 3 plus-signs to be arranged. There are $C^{11}_3 = 165$ solutions.
    \item[b)] Since $a_1, a_2, a_3$ are integers, we can rewrite the equation as $a_1+ a_2 + a_3 = 12$. Consider a new variable $x$ that is a non-negative integer. We can add it to the left side to create:\\
    $a_1 + a_2 + a_3 + x = 12$. This is valid because $x$ will take a value 0 to 12, leaving 12 to 0 for the sum $a_1 + a_2 + a_3$.\\
    There are 12 ones and 3 plus-signs to be arranged. There are $C^{15}_3 = 455$ solutions.
    \item[c)] Define a variable $b_1 = a_1 - 6$, where $b_1$ is a non-negative integer. Subtracting 5 from both sides and substituting in $b_1$ for $a_1$, the equation becomes $b_1 + a_2 + a_3 + a_4 + a_5 \leq 6$. Define another variable $x$ that is a non-negative integer. We can add it to the left side to create:\\
    $b_1 + a_2 + a_3 + a_4 + a_5 + x = 6$. This is valid because $x$ will take a value 0 to 6, leaving 6 to 0 for the sum $b_1 + a_2 + a_3 + a_4 + a_5$. Note how the $a_3 \leq 10$ constraint is always satisfied.\\
    There are 6 ones and 5 plus-signs to be arranged. There are $C^{11}_5 = 462$ solutions.
\end{enumerate}
\end{solution}

\item How many different strings can be created by rearranging the character string ``i am out of ideas"? Ignore spaces. \pt{8}
\begin{solution}
Character occurrences: 2 of a, i, o; 1 of d, e, f, m, s, t, u

$\frac{13!}{2!2!2!} = 778,377,600$ arrangements
\end{solution}

\item CS 2050 has 9 recitation sections. 6 of the recitation sections are led by a pair of TAs, and 3 recitation sections are led by a trio of TAs. There are also 5 TAs who do not teach recitation. How many ways are there to line up all of the TAs such that each TA with recitation partner(s) is standing next to them? \pt{5}
\begin{solution}
Let there be 14 ``units": the 9 recitation groups and the 5 individual TAs. There are $14!$ ways to order the units.\\
In addition, we need to consider order within the units. There are 6 units of two ($2!$ arrangements per unit) and 3 units of three ($3!$ arrangements per unit).\\ Multiplying the unit and inter-unit arrangements gives: $14! (2!)^6 (3!)^3 = 1,205,152,697,548,800$ arrangements
\end{solution}

\item How many eight digit numbers contain the digit 2 once, the digit 3 twice, and the digit 4 thrice. Leading 0s are allowed. \pt{5}
\begin{solution}
$(C^8_1 \cdot C^7_2 \cdot C^5_3) \cdot (C^7_1)^2= 82,320$
\end{solution}

\item How many eight digit numbers contain the digit 2 once, the digit 3 twice, the digit 4 thrice, and more 5s than threes? Leading 0s are allowed. \pt{4}
\begin{solution}
0 numbers satisfy these conditions. $1 + 2 + 3 = 6$ digits are taken by 2s, 3s, and 4s. There are $8 - 6 = 2$ digits remaining, so there cannot be more 5s than 3s.
\end{solution}

\item A lattice path in the plane is a sequence of ordered pairs of integers $(m_1,n_1),(m_2,n_2),\dots,(m_t,n_t)$ such that for all $i < t-1$: \pte{5}
\begin{enumerate}
    \item[i.] $m_{i+1}=m_i+1$ and $n_{i+1} = n_i$ or
    \item[ii.] $m_{i+1}=m_i$ and $n_{i+1}=n_i+1$
\end{enumerate}

\begin{enumerate}
    \item How many lattice paths exist from $(0,0)$ to $(11,13)$?
    \item How many lattice paths exist from $(5,3)$ to $(11,13)$?
    \item How many lattice paths exist from $(0,0)$ to $(11,13)$ that pass through $(5,3)$?
    \item How many lattice paths exist from $(0,0)$ to $(11,13)$ that do not pass through $(5,3)$?
\end{enumerate}
\begin{solution}
\begin{enumerate}
    \item There are 11 `rights' and 13 `ups' that are to be ordered. With 24 total moves, there are $C^{24}_{11} = 2,496,144$ paths.
    \item There are 6 `rights' and 10 `ups' that are to be ordered. With 16 total moves, there are $C^{16}_{6} = 8,008$ paths.
    \item To get to (5, 3) There are 5 `rights' and 3 `ups' that are to be ordered, which is $C^{8}_{5} = 56$ paths. To get from (5, 3) to (11, 13) there are $8,008$ paths (see (b)). Combining the two parts, there are $56 \cdot 8,008 = 448,448$ paths.
    \item There are $2,496,144$ total paths from (0, 0) to (11, 13). We then subtract the $448,448$ paths that pass through (5, 3). There are $2,496,144 - 448,448 = 2,047,696$ paths.
\end{enumerate}
\end{solution}

\end{enumerate}
\end{document}

