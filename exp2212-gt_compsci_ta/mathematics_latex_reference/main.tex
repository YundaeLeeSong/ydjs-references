%%%%%%%%%%%%%%%%%%%%%%%%%%%%%%%%%%%%%%%%%%%%%%%%%%%%%%%%%%%%%%%%%%%%%%%%%%%%%%
% Preamble (Configuration)
%%%%%%%%%%%%%%%%%%%%%%%%%%%%%%%%%%%%%%%%%%%%%%%%%%%%%%%%%%%%%%%%%%%%%%%%%%%%%%
\documentclass{article}
\usepackage[utf8]{inputenc}
%%%%%%%%%%%%%%%%%%%%%%%%%%%%%%%%%%%%%%%%%%%%%%%%%%%%%%%%%%%%%%%%%%%%%%%%%%%%%%
% Mathematics
%%%%%%%%%%%%%%%%%%%%%%%%%%%%%%%%%%%%%%%%%%%%%%%%%%%%%%%%%%%%%%%%%%%%%%%%%%%%%%
\usepackage{mathtools}
\usepackage{amsthm}
\usepackage{amsmath}
\usepackage{amssymb}
\renewcommand\qedsymbol{$\blacksquare$}
%%%%%%%%%%%%%%%%%%%%%%%%% Custom Macro %%%%%%%%%%%%%%%%%%%%%%%%%
\newtheorem{theorem}{Theorem}[section] % theorem
\newtheorem{corollary}{Corollary}[theorem]
\newtheorem{lemma}[theorem]{Lemma}
\theoremstyle{definition}
\newtheorem{definition}{Definition}[section]
\theoremstyle{remark}
\newtheorem*{remark}{Remark}



%%%%%%%%%%%%%%%%%%%%%%%%%%%%%%%%%%%%%%%%%%%%%%%%%%%%%%%%%%%%%%%%%%%%%%%%%%%%%%
% Text (Document)
%%%%%%%%%%%%%%%%%%%%%%%%%%%%%%%%%%%%%%%%%%%%%%%%%%%%%%%%%%%%%%%%%%%%%%%%%%%%%%
\begin{document}
\section{Workspace}




$$\neg (\exists x P(x)) $$
$$\equiv \neg (P({x}_{1}) \lor P({x}_{2}) \lor ... \lor P({x}_{n}))$$
$$\equiv \neg P({x}_{1}) \land \neg P({x}_{2}) \land ... \land \neg P({x}_{n})$$
$$\equiv \forall x \neg P(x)$$





\subsection{sample01}
We want to show that $-2 = 2$.    
\begin{proof}
$\arraycolsep=1.5pt
\begin{array}[t]{rl|l}
-2     &= 2   & \text{ assuming the conclusion}\\
(-2)^2 &= 2^2 & \text{ square both sides} \\
4      &= 4   & \text{ as desired}
\end{array}$
\end{proof}

\subsection{sample02}
We want to show that $-2=2$.
\begin{proof}
    \begin{align*}
        -2 &= 2 && \smash{\Big|}\text{ assuming the conclusion} \\
        (-2)^2 &= 2^2 && \smash{\Big|}\text{ square both sides} \\
        4 &= 4 && \smash{\Big|}\text{ as desired}
    \end{align*}
\end{proof}



\subsection{Equation and Theorem}



        \begin{equation}\label{eq1}\begin{split}
            A & = \frac{\pi r^2}{2} \\
            & = \frac{1}{2} \pi r^2
        \end{split}\end{equation}

        
        \begin{equation}\label{eq2}\begin{split}
            A\cup(B\cup A) & =B \cup(A\cup A) \\
            & = B \cup A
        \end{split}\end{equation}
        %% https://www.overleaf.com/learn/latex/Aligning_equations_with_amsmath
        %% https://tex.stackexchange.com/questions/40492/what-are-the-differences-between-align-equation-and-displaymath


\begin{theorem}
Let \(f\) be a function whose derivative exists in every point, then \(f\) 
is a continuous function.
\end{theorem}








\subsection{Proof}

    \begin{tabular}{ll|l}
    Step & Statement & Reasoning \\ \hline
    (1) & $n = 2k + 1$ for some $k \in \mathbf{Z}$ & definition of odd\\
    (2) & $n^3 = (2k+1)^3$ & cube both sides of (1)\\
    (3) & $n^3 = 8k^3 + 12k^2 + 6k + 1$ & expand (2)\\
    (4) & $n^3 + 12 = 8k^3 + 12k^2 + 6k + 13$ & add 12 to both sides\\
    (5) & $n^3 = 2(4k^3 + 6k^2 + 3k + 6) + 1$ & factor out 2 from RHS\\
    (6) & $t = 4k^3 + 6k^2 + 3k + 6 t \in \mathbf{Z}$ & Closure of multiplication and addition of $\mathbf{Z}$ \\
    (7) & $n^3 + 12 = 2t + 1$ & Substitution of (6) into (5)\\
    (8) & $n^3 + 12$ is odd & Definition of odd\\
    \end{tabular}






\subsection{Theorem Examples}
Theorems can easily be defined:

\begin{theorem}
Let \(f\) be a function whose derivative exists in every point, then \(f\) is 
a continuous function.
\end{theorem}

\begin{theorem}[Pythagorean theorem]
\label{pythagorean}
This is a theorem about right triangles and can be summarised in the next 
equation 
\[ x^2 + y^2 = z^2 \]
\end{theorem}

And a consequence of theorem \ref{pythagorean} is the statement in the next 
corollary.
\begin{corollary}
There's no right rectangle whose sides measure 3cm, 4cm, and 6cm.
\end{corollary}

You can reference theorems such as \ref{pythagorean} when a label is assigned.
\begin{lemma}
Given two line segments whose lengths are \(a\) and \(b\) respectively there is a 
real number \(r\) such that \(b=ra\).
\end{lemma}



\subsection{Definition Examples}
Unnumbered theorem-like environments are also possible.
\begin{remark}
This statement is true, I guess.
\end{remark}
And the next is a somewhat informal definition
\begin{definition}[Fibration]
A fibration is a mapping between two topological spaces that has the homotopy lifting property for every space \(X\).
\end{definition}









\subsection{Proof}
\begin{lemma}
Given two line segments whose lengths are \(a\) and \(b\) respectively there 
is a real number \(r\) such that \(b=ra\).
\end{lemma}
\begin{proof}
To prove it by contradiction try and assume that the statement is false,
proceed from there and at some point you will arrive to a contradiction.
\end{proof}









\subsection{Quick Reference}
\subsubsection{Fraction}



The binomial coefficient, \(\binom{n}{k}\), is defined by the expression:
\[
    \binom{n}{k} = \frac{n!}{k!(n-k)!}
\]


Fractions can be used inline within the paragraph text, for 
example $\frac{1}{2}$, or displayed on their own line, 
such as this:
$$\frac{1}{2}$$



We use the \texttt{amsmath} package command
\verb|\text{...}| to create text-only fractions
like this:

$$\frac{\text{numerator}}{\text{denominator}}$$

Without the \verb|\text{...}| command the result 
looks like this:

$$\frac{numerator}{denominator}$$



\subsubsection{Fraction within a Paragraph}

Fractions typeset within a paragraph typically look like this: \(\frac{3x}{2}\). You can force \LaTeX{} to use the larger display style, such as \( \displaystyle \frac{3x}{2} \), which also has an effect on line spacing. The size of maths in a paragraph can also be reduced: \(\scriptstyle \frac{3x}{2}\) or \(\scriptscriptstyle \frac{3x}{2}\). For the \verb|\scriptscriptstyle| example note the reduction in spacing: characters are moved closer to the \textit{vinculum} (the line separating numerator and denominator).

Equally, you can change the style of mathematics normally typeset in display style:

\[f(x)=\frac{P(x)}{Q(x)}\quad \textrm{and}\quad \textstyle f(x)=\frac{P(x)}{Q(x)}\quad \textrm{and}\quad \scriptstyle f(x)=\frac{P(x)}{Q(x)}\]



\subsubsection{Nested Fraction}

Fractions can be nested but, in this example, note how the default math styles, as used in the denominator, don't produce ideal results...

\[ \frac{1+\frac{a}{b}}{1+\frac{1}{1+\frac{1}{a}}} \]

\noindent ...so we use \verb|\displaystyle| to improve typesetting:

\[ \frac{1+\frac{a}{b}} {\displaystyle 1+\frac{1}{1+\frac{1}{a}}} \]

Here is an example which uses the \texttt{amsmath} \verb|\cfrac| command:

\[
  a_0+\cfrac{1}{a_1+\cfrac{1}{a_2+\cfrac{1}{a_3+\cdots}}}
\]

Here is another example, derived from the \texttt{amsmath} documentation, which demonstrates left
and right placement of the numerator using \verb|\cfrac[l]| and \verb|\cfrac[r]| respectively:
\[
\cfrac[l]{1}{\sqrt{2}+
\cfrac[r]{1}{\sqrt{2}+
\cfrac{1}{\sqrt{2}+\dotsb}}}
\]







%% https://www.overleaf.com/learn/latex/Fractions_and_Binomials





\end{document}
