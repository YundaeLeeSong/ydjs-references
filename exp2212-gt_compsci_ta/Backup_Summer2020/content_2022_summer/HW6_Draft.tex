\begin{enumerate}




    %%%%%%%%%%%%%%%%%%%%%%%%%%%%%%%%%%%%%%%%%%%%%%%%%%%%%%%%%%%%%%%%
    % Question #1
    %%%%%%%%%%%%%%%%%%%%%%%%%%%%%%%%%%%%%%%%%%%%%%%%%%%%%%%%%%%%%%%%

    %%%%%%%%%%%%%%%%%%%%%%%%%%%% Rubric %%%%%%%%%%%%%%%%%%%%%%%%%%%%








    %%%%%%%%%%%%%%%%%%%%%%%%%%%%%%%%%%%%%%%%%%%%%%%%%%%%%%%%%%%%%%%%
    % Question #2
    %%%%%%%%%%%%%%%%%%%%%%%%%%%%%%%%%%%%%%%%%%%%%%%%%%%%%%%%%%%%%%%%

    %%%%%%%%%%%%%%%%%%%%%%%%%%%% Rubric %%%%%%%%%%%%%%%%%%%%%%%%%%%%





    %%%%%%%%%%%%%%%%%%%%%%%%%%%%%%%%%%%%%%%%%%%%%%%%%%%%%%%%%%%%%%%%
    % Question #3
    %%%%%%%%%%%%%%%%%%%%%%%%%%%%%%%%%%%%%%%%%%%%%%%%%%%%%%%%%%%%%%%%

    %%%%%%%%%%%%%%%%%%%%%%%%%%%% Rubric %%%%%%%%%%%%%%%%%%%%%%%%%%%%





    %%%%%%%%%%%%%%%%%%%%%%%%%%%%%%%%%%%%%%%%%%%%%%%%%%%%%%%%%%%%%%%%
    % Question #4
    %%%%%%%%%%%%%%%%%%%%%%%%%%%%%%%%%%%%%%%%%%%%%%%%%%%%%%%%%%%%%%%%

    %%%%%%%%%%%%%%%%%%%%%%%%%%%% Rubric %%%%%%%%%%%%%%%%%%%%%%%%%%%%





    %%%%%%%%%%%%%%%%%%%%%%%%%%%%%%%%%%%%%%%%%%%%%%%%%%%%%%%%%%%%%%%%
    % Question #5
    %%%%%%%%%%%%%%%%%%%%%%%%%%%%%%%%%%%%%%%%%%%%%%%%%%%%%%%%%%%%%%%%

    %%%%%%%%%%%%%%%%%%%%%%%%%%%% Rubric %%%%%%%%%%%%%%%%%%%%%%%%%%%%





    %%%%%%%%%%%%%%%%%%%%%%%%%%%%%%%%%%%%%%%%%%%%%%%%%%%%%%%%%%%%%%%%
    % Question #6
    %%%%%%%%%%%%%%%%%%%%%%%%%%%%%%%%%%%%%%%%%%%%%%%%%%%%%%%%%%%%%%%%

    %%%%%%%%%%%%%%%%%%%%%%%%%%%% Rubric %%%%%%%%%%%%%%%%%%%%%%%%%%%%





    %%%%%%%%%%%%%%%%%%%%%%%%%%%%%%%%%%%%%%%%%%%%%%%%%%%%%%%%%%%%%%%%
    % Question #7
    %%%%%%%%%%%%%%%%%%%%%%%%%%%%%%%%%%%%%%%%%%%%%%%%%%%%%%%%%%%%%%%%

    %%%%%%%%%%%%%%%%%%%%%%%%%%%% Rubric %%%%%%%%%%%%%%%%%%%%%%%%%%%%





    %%%%%%%%%%%%%%%%%%%%%%%%%%%%%%%%%%%%%%%%%%%%%%%%%%%%%%%%%%%%%%%%
    % Question #8
    %%%%%%%%%%%%%%%%%%%%%%%%%%%%%%%%%%%%%%%%%%%%%%%%%%%%%%%%%%%%%%%%

    %%%%%%%%%%%%%%%%%%%%%%%%%%%% Rubric %%%%%%%%%%%%%%%%%%%%%%%%%%%%





    %%%%%%%%%%%%%%%%%%%%%%%%%%%%%%%%%%%%%%%%%%%%%%%%%%%%%%%%%%%%%%%%
    % Question #9
    %%%%%%%%%%%%%%%%%%%%%%%%%%%%%%%%%%%%%%%%%%%%%%%%%%%%%%%%%%%%%%%%

    %%%%%%%%%%%%%%%%%%%%%%%%%%%% Rubric %%%%%%%%%%%%%%%%%%%%%%%%%%%%





    
\item Use mathematical induction to prove that $n^3 +5n -12$ is divisible by $6$ for all positive integers. \pt{20}

\begin{rubric}
-2, missing introduction (proof technique, propositions)

-6, worked backwards from $P(k+1)$

-6, missing basis step

-4, incorrect basis step

-2, missing basis step conclusion

-4, did not assume $P(k)$

-10, assumed $P(k+1)$

-4, did not prove inductive conditional

-2, missing inductive step conclusion

-2, no variable domains

-2, missing final conclusion
\end{rubric}

\item Use mathematical induction to show that $2^n < n!$ for all $n \geq 4$. \pt{20}

\begin{rubric}
-2, missing introduction (proof technique, propositions)

-6, worked backwards from $P(k+1)$

-6, missing basis step

-4, incorrect basis step

-2, missing basis step conclusion

-4, did not assume $P(k)$

-10, assumed $P(k+1)$

-4, did not prove inductive conditional

-2, missing inductive step conclusion

-2, no variable domains

-2, missing final conclusion
\end{rubric}

\item Use strong induction to prove that $\sqrt{12}$ is irrational. (\textbf{Hint:} Let $P(n)$ be the statemement that $\sqrt{2} \neq \frac{n}{b}$, $b \in \mathbb{Z}^+$) \pt{20}

\begin{rubric}
-2, missing introduction (proof technique, propositions)

-6, worked backwards from $P(k+1)$

-6, missing basis step

-4, incorrect basis step

-2, missing basis step conclusion

-4, did not assume $P(k)$

-10, assumed $P(k+1)$

-4, did not prove inductive conditional

-2, missing inductive step conclusion

-2, no variable domains

-2, missing final conclusion
\end{rubric}

\item Use strong induction to prove that every integer amount of postage greater than 14 cents can be formed using only 3-cent and 8-cent coins. \pt{20}

\begin{rubric}
-2, missing introduction (proof technique, propositions)

-6, worked backwards from $P(k+1)$

-6, missing basis step

-4, incorrect basis step

-2, missing basis step conclusion

-4, did not assume $P(k)$

-10, assumed $P(k+1)$

-4, did not prove inductive conditional

-2, missing inductive step conclusion

-2, no variable domains

-2, missing final conclusion
\end{rubric}

\item Given an initial base case of $f(0)=1$, find $f(2), f(3)$ and $f(4)$ for the following functions: \pte{5}
\begin{enumerate}
    \item $f(n + 1) = nf(n)$
    \item $f(n + 1) = 2^{f(n)}$
\end{enumerate}

\begin{rubric}
-2, arithmetic error

-3, invalid input for $f(1)$; error propagation
\end{rubric}

\item Recursively define each of the following sets: \pte{5}
\begin{enumerate}
    \item The set of perfect squares.
    \item The set of integers congruent to 4 modulo 5.
\end{enumerate}

\begin{rubric}
-2, incorrect base case

-3, incorret recursive definition

-1, bounding error (indexed domain incorrectly)
\end{rubric}

\end{enumerate}