\begin{enumerate}



    %%%%%%%%%%%%%%%%%%%%%%%%%%%%%%%%%%%%%%%%%%%%%%%%%%%%%%%%%%%%%%%%
    % Question #1
    %%%%%%%%%%%%%%%%%%%%%%%%%%%%%%%%%%%%%%%%%%%%%%%%%%%%%%%%%%%%%%%%

    %%%%%%%%%%%%%%%%%%%%%%%%%%%% Rubric %%%%%%%%%%%%%%%%%%%%%%%%%%%%








    %%%%%%%%%%%%%%%%%%%%%%%%%%%%%%%%%%%%%%%%%%%%%%%%%%%%%%%%%%%%%%%%
    % Question #2
    %%%%%%%%%%%%%%%%%%%%%%%%%%%%%%%%%%%%%%%%%%%%%%%%%%%%%%%%%%%%%%%%

    %%%%%%%%%%%%%%%%%%%%%%%%%%%% Rubric %%%%%%%%%%%%%%%%%%%%%%%%%%%%





    %%%%%%%%%%%%%%%%%%%%%%%%%%%%%%%%%%%%%%%%%%%%%%%%%%%%%%%%%%%%%%%%
    % Question #3
    %%%%%%%%%%%%%%%%%%%%%%%%%%%%%%%%%%%%%%%%%%%%%%%%%%%%%%%%%%%%%%%%

    %%%%%%%%%%%%%%%%%%%%%%%%%%%% Rubric %%%%%%%%%%%%%%%%%%%%%%%%%%%%





    %%%%%%%%%%%%%%%%%%%%%%%%%%%%%%%%%%%%%%%%%%%%%%%%%%%%%%%%%%%%%%%%
    % Question #4
    %%%%%%%%%%%%%%%%%%%%%%%%%%%%%%%%%%%%%%%%%%%%%%%%%%%%%%%%%%%%%%%%

    %%%%%%%%%%%%%%%%%%%%%%%%%%%% Rubric %%%%%%%%%%%%%%%%%%%%%%%%%%%%





    %%%%%%%%%%%%%%%%%%%%%%%%%%%%%%%%%%%%%%%%%%%%%%%%%%%%%%%%%%%%%%%%
    % Question #5
    %%%%%%%%%%%%%%%%%%%%%%%%%%%%%%%%%%%%%%%%%%%%%%%%%%%%%%%%%%%%%%%%

    %%%%%%%%%%%%%%%%%%%%%%%%%%%% Rubric %%%%%%%%%%%%%%%%%%%%%%%%%%%%





    %%%%%%%%%%%%%%%%%%%%%%%%%%%%%%%%%%%%%%%%%%%%%%%%%%%%%%%%%%%%%%%%
    % Question #6
    %%%%%%%%%%%%%%%%%%%%%%%%%%%%%%%%%%%%%%%%%%%%%%%%%%%%%%%%%%%%%%%%

    %%%%%%%%%%%%%%%%%%%%%%%%%%%% Rubric %%%%%%%%%%%%%%%%%%%%%%%%%%%%





    %%%%%%%%%%%%%%%%%%%%%%%%%%%%%%%%%%%%%%%%%%%%%%%%%%%%%%%%%%%%%%%%
    % Question #7
    %%%%%%%%%%%%%%%%%%%%%%%%%%%%%%%%%%%%%%%%%%%%%%%%%%%%%%%%%%%%%%%%

    %%%%%%%%%%%%%%%%%%%%%%%%%%%% Rubric %%%%%%%%%%%%%%%%%%%%%%%%%%%%





    %%%%%%%%%%%%%%%%%%%%%%%%%%%%%%%%%%%%%%%%%%%%%%%%%%%%%%%%%%%%%%%%
    % Question #8
    %%%%%%%%%%%%%%%%%%%%%%%%%%%%%%%%%%%%%%%%%%%%%%%%%%%%%%%%%%%%%%%%

    %%%%%%%%%%%%%%%%%%%%%%%%%%%% Rubric %%%%%%%%%%%%%%%%%%%%%%%%%%%%





    %%%%%%%%%%%%%%%%%%%%%%%%%%%%%%%%%%%%%%%%%%%%%%%%%%%%%%%%%%%%%%%%
    % Question #9
    %%%%%%%%%%%%%%%%%%%%%%%%%%%%%%%%%%%%%%%%%%%%%%%%%%%%%%%%%%%%%%%%

    %%%%%%%%%%%%%%%%%%%%%%%%%%%% Rubric %%%%%%%%%%%%%%%%%%%%%%%%%%%%





    
\item Given the sets: 
\begin{align*}
    A &:= \{1,8,27,64,125\}\\
    B &:= \{1,16,81,256,635\}\\
    C &:= \{x^2 | x \in \mathbb{Z}\}\
\end{align*}
Find each of the following. You may use set builder notation if necessary.
\pte 5

\begin{enumerate}
    \item $B \cap C$
    \item $A-(B-C)$
    \item $A \cap \emptyset$
    \item $(A \cap B) \times B$
    \item $|\mathcal{P}(B\cap C)|$
\end{enumerate}

\begin{rubric}
-1, per missing element

-2, per incorrect element

\end{rubric}

\begin{solution}
\begin{enumerate}
    \item \{1, 16, 81, 256\}
    \item \{1, 8, 27, 64, 125\}
    \item \{ \}
    \item \{(1, 1), (1, 16), (1, 81), (1, 256), (1, 635)\}
    \item 16
\end{enumerate}
\end{solution}

\item Determine whether the following statements are true or false. \pte{3}

\begin{enumerate}
    \item $\emptyset \in \emptyset$
    \item $\emptyset \subset \emptyset$
    \item $\{\emptyset\} \in \{\emptyset,\{\emptyset\}\}$
    \item $|\{\{\emptyset\}\}| \leq |\{\{\emptyset\}, \{\emptyset\}\}|$
    \item $\mathcal{P}(\emptyset) \subseteq \emptyset$
\end{enumerate}

\begin{rubric}
-3, AON
\end{rubric}

\begin{solution}
\begin{enumerate}
    \item F
    \item F
    \item T
    \item T
    \item F
\end{enumerate}
\end{solution}

\item Given sets $A$, $B$, and $C$, prove the first distributive law for sets using logical equivalences:   \pt{10}
\begin{center}
    $A \cup (B \cap C) = (A \cup B) \cap (A \cup C)$
\end{center}

\begin{rubric}
-5, fails to reference conjunction or disjunction (and/or)

-5, missing direction ($x \in $ LHS or $x \in $ RHS) for set equality

-1, per skipped/missing steps

-2, missing conclusion

\end{rubric}

\begin{solution}\\
Case 1 ($x \in A \cup (B \cap C)$):\\
\begin{tabular}{c|c}
    Statement & Reasoning \\
    \hline
    $x \in A \cup (B \cap C)$ & LHS \\
    $x \in A \lor x \in (B \cap C)$ & Definition of union \\
    $x \in A \lor (x \in B \land x \in C)$ & Definition of intersection \\
    $(x \in A \lor x \in B) \land (x \in A \lor x \in C)$ & Distributive law \\
    $x \in (A \cup B) \land x \in (A \cup C)$ & Definition of union \\
    $x \in (A \cup B) \cap (A \cup C)$ & Definition of intersection \\
    
\end{tabular}\\
Case 2 ($x \in (A \cup B) \cap (A \cup C)$):\\
\begin{tabular}{c|c}
    Statement & Reasoning \\
    \hline
    $x \in (A \cup B) \cap (A \cup C)$ & RHS \\
    $x \in (A \cup B) \land x \in (A \cup C)$ & Definition of intersection \\
    $(x \in A \lor x \in B) \land (x \in A \lor x \in C)$ & Definition of union \\
    $x \in A \lor (x \in B \land x \in C)$ & Distributive law \\
    $x \in A \lor x \in (B \cap C)$ & Definition of intersection \\
    $x \in A \cup (B \cap C)$ & Definition of union\\
\end{tabular}

We have proven that the statement holds in each direction. Thus, $A \cup (B \cap C) = (A \cup B) \cap (A \cup C)$. \qedsymbol

\end{solution}

\item Let $f$ and $g$ be functions such that: \pte{5}
\begin{align*}
    f(x) &= \lceil x \rceil, \text{where } f : \mathbb{R} \rightarrow \mathbb{Z}\\
    g(x) &= x+1, \text{where } g : \mathbb{Z} \rightarrow \mathbb{R}
 \end{align*}

Determine whether the following function compositions are one-to-one, onto, both, or neither. Justify your reasoning for full credit.

\begin{enumerate}
    \item $(f \circ g)(x)$
    \item $(g \circ f)(x)$
\end{enumerate}

\begin{rubric}
    -2, doesn't check one to one
    
    -2, doesn't check onto
    
    -2 fails to reference function domains
\end{rubric}

\begin{solution}

    (a) Onto and one-to-one, \mathbb{Z} \rightarrow \mathbb{Z}\\
    (b)\ Neither, \mathbb{R} \rightarrow \mathbb{R}

\end{solution}

\item For each of the following functions from $\mathbb{R}^+ \rightarrow \mathbb{R}$, find the least integer $n$ such that $f(x)$ is $O(x^n)$ if possible. If not, explain why the function is not $O(x^n)$ for any $n$. \pte{5}

\begin{enumerate}
    \item $f(x) = x^{\sqrt{2}}$
    \item $f(x) = 3x^2 + x^3\log(3x^2)$
    \item $f(x) = \frac{x^5 + x^4}{x^9 + x^6\log9x}$
    \item $f(x) = \log x^{\log x}$
\end{enumerate}

\begin{solution}
\begin{enumerate}
    \item 2
    \item 4
    \item -4
    \item 2
\end{enumerate}
\end{solution}

\item Use sequence notation to represent each of the following sequences in two ways: as a closed form expression, and then as a recurrence relation (and initial condition). \pte{2}

\begin{enumerate}
    \item 9, 12, 15, 18, 21...
    \item 1, 1, 1, 1, 1...
    \item 3, 15, 75, 375, 1875...
\end{enumerate}

\begin{rubric}
    -2 each, AON
\end{rubric}

\begin{solution}
\begin{enumerate}
    \item Closed form: $a_{n}=9+3n, n=0,1,2...$\\ Recurrence: $a_{n}=a_{n-1}+3, a_{0}=9, n=1,2,3...$
    \item Closed form: $a_{n} = 1, n=0,1,2...$\\ Recurrence: $a_{n}=a_{n}+0, a_{0}=1,n=0,1,2...$
    \item Closed form: $a_{n}=3*5^{n}$\\ Recurrence: $a_{n}=a_{n-1}*5, a_{0}=3, n=1,2,3...$
\end{enumerate}
\end{solution}

\item Evaluate the following sums. Show your work by expanding the terms. \pte{2}

\begin{enumerate}
    \item $\sum_{n=1}^{4} \frac{n(n-1)}{n}$
    \item $\sum_{x=1}^{3} 3^x$
    \item $\sum_{j=0}^{2} \sum_{i=0}^{j} 2^{-i}$
\end{enumerate}

\begin{rubric}
    -2, did not expand terms
    
    -2, incorrect
    
    -1, arithmetic error
\end{rubric}

\begin{solution}
\begin{enumerate}
    \item 6
    \item 39
    \item $\frac{17}{4}$ or 4.25
\end{enumerate}
\end{solution}

\item Give pseudocode for finding the average of the positive numbers in a list. \pt{4}

\begin{rubric}
-2, invalid pseudocode

-2, algorithm is slower than $O(n)$
\end{rubric}

\begin{solution}
    \\ Function FindAverage($a_{0}...a_{n}$:integers):\\ \ . \hspace{1pt} sum= 0;\\
    . \hspace{1pt} positiveCount= 0;\\
    . \hspace{3pt} for i := 0 \rightarrow n \\ 
    . \hspace{20pt} if ($a_{i} > 0$): \\
    . \hspace{30pt} sum += $a_{i}$ \\
    . \hspace{30pt} positiveCount++;\\
    . \hspace{1pt} if (positiveCount == 0):\\
    . \hspace{5pt} return 0\\
    . \hspace{1pt} return sum/positiveCount
    
\end{solution}

\item Consider a function that, given a list of integers, first computes the sum of the elements, then the product, and then averages those two numbers together. The function skips the first element of each list during every iteration by starting the index at 1 outside of the loop. \pte{4}
\begin{enumerate}
    \item Give pseudocode for this algorithm.
    \item What is the runtime of your algorithm?
    \item Given the list $l = [5,3,8,9,2,3,1]$, what will the be output?
\end{enumerate}

\begin{rubric}
-2, incorrect indexing (bounding error)

-4, invalid pseudocode

-4, algorithm is slower than $O(n)$

-4, arithmetic error
\end{rubric}

\begin{solution}
\begin{enumerate}
    \item 
    \begin{algorithmic}
    \State $\text{sum} \gets 0$
    \State $\text{product} \gets 0$
    \State $i \gets 1$
    \While{$i < len(range(l))$}
        \State $\text{sum} \gets \text{sum} + l_{i}$
        \State $\text{product} \gets \text{product} \times l_{i}$
        \State $i \gets i + 1$
        \EndWhile \\
    \textbf{return} $\frac{sum+product}{2}$
\end{algorithmic}

    \item O(n)
    \item 661
\end{enumerate}
\end{solution}

\item (Extra Credit) Let $A$, $B$, and $C$ be sets. The inclusion-exclusion principle for two sets states that $|A \cup B| = |A| + |B| - |A \cap B|$. Use the inclusion-exclusion principle for two sets to find $|A \cup B \cup C|.$ \pt{1}

\begin{solution}\\
\begin{tabular}{c|c}
    Statement & Reasoning \\
    \hline
    $S = (B \cup C)$ & Instantiate new set $S$ \\
    $|A \cup S| = |A| + |S| - |A \cap S|$ & Inclusion-Exclusion Principle for $n = 2$ \\
    $|A \cup (B \cup C)| = |A| + |(B \cup C)| - |A \cap (B \cup C)|$ & Substitute $S$ \\
    $|A \cup B \cup C| = |A| + |(B \cup C)| - |A \cap (B \cup C)|$ & Associative Law \\
    $|A \cup B \cup C| = |A| + |B| + |C| - |B \cap C| - |A \cap (B \cup C)|$ & Inclusion-Exclusion Principle for $n = 2$ \\
    $|A \cup B \cup C| = |A| + |B| + |C| - |B \cap C| - |A \cap B| - |A \cap C| - |A \cap B \cap C|$ & Distributive Property \\
\end{tabular}
\end{solution}

\item (Extra Credit) Let $S$ be the set containing all sets which do not contain themselves. Does $S$ contain itself? Show that the answer to this questions leads to a contradiction. \pt{1}

\begin{solution}
If $S \notin S$, then $S \in S$ by definition. However, if $S \in S$, then $S$ may not contain $S$ by the definition of $S$. Thus, both cases lead to a contradiction and we have shown that there does not exist a set which may contain itself.
\end{solution}

\item (Extra Credit) Determine where there exists a bijection $f : \mathbb{Z} \rightarrow \mathbb{Z}^+$. \pt{1}

\begin{solution}\\
Accept any function that splits $\mathbb{Z}$ and maps positives to evens and negatives to odds, or else. E.g.
\[
    f(x)= 
\begin{cases}
    2x,& \text{if } x\geq 0\\
    -2x-1,              & \text{otherwise}
\end{cases}
\]
\end{solution}

\end{enumerate}