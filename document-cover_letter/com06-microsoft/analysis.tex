% One-page, ATS-friendly LaTeX resume + cover letter (Automation Genesis — Discrete Engineering Track)
% Replace all bracketed placeholders [like this] with your personal details.
\documentclass[11pt]{article}
\hbadness=10000                       % Suppress Warnings (Horizontal)
\vbadness=10000                       % Suppress Warnings (Vertical)
\usepackage[margin=0.7in]{geometry}
\usepackage{parskip}          % better spacing between paragraphs
\usepackage{enumitem}         % control list spacing
\usepackage[hidelinks]{hyperref}
\usepackage{helvet}

% Compact list spacing
\setlist[itemize]{leftmargin=*,itemsep=2pt,topsep=2pt}
\setlist[enumerate]{leftmargin=*,itemsep=2pt,topsep=2pt}

\begin{document}


\begin{tabular}{|l|l|l|}
\hline Salesforce Title & Common Equivalent & Notes \\
\hline Software Engineer (SWE) & Entry-Level / Junior Engineer & Usually for new grads or early-career (0-2 yrs). Look for this if you're just starting out. \\
\hline Associate MTS (AMTS) & Entry-Level SWE & "MTS" stands for Member of Technical Staff. "Associate" is the lowest level-ideal for entry roles. \\
\hline MTS (Member of Technical Staff) & Mid-Level SWE & Requires 2-5 yrs of experience; not entry level. \\
\hline Senior MTS (SMTS) & Senior SWE & Experienced engineer, usually 5-8 yrs+. \\
\hline Lead MTS (LMTS) & Staff / Tech Lead & High-level individual contributor, typically 8-10 yrs+. \\
\hline Principal MTS (PMTS) & Principal Engineer & Top-tier IC, often 10+ yrs or recognized expert. \\
\hline
\end{tabular}


\section*{Software Engineer - Promotions Services at T-Mobile}

The \textbf{Software Engineer} works with a team of other software engineers, 
network and systems engineers to design, implement, and deploy software which 
meet customer's requirements, scales easily, removes the limitations of 
traditional networking solutions, and supports deployment in highly available 
environments. The Software Engineer participates and leads in architecture 
and design of various software components passionate about crafting 
applications that leverage technologies such as virtualization, micro 
services, SDN, NFV, and Big Data platforms and technologies.

\section*{Software Engineer by \textbf{T-Mobile USA, Inc.}}
\textbf{Job ID:} REQ326194. This role focuses on the design, implementation, 
and deployment of software for T-Mobile's internal and customer-facing 
services, specifically related to the Promotions Services team.

\begin{itemize}
  \item \textbf{Field:} Information Technology / Software Engineering
  \item \textbf{Pay range:} \$92,500 -- \$166,800 annually (plus 15\% corporate bonus target; 
    actual offer depends on location, qualifications, and experience).
  \item \textbf{Benefits:} Health, dental, and vision insurance; 401(k); 
    employee stock grants; employee stock purchase plan (ESPP); paid time off 
    (approx. 4 weeks for new full-time hires); paid parental leave; 
    tuition assistance. Details referenced at: 
    \url{www.t-mobilebenefits.com}
  \item \textbf{Employment type:} Full-time, permanent
  \item \textbf{Locations:} Frisco, TX; Atlanta, GA
  \item Posted on \textbf{2025-10-14}
  \begin{enumerate}[label=\arabic*., leftmargin=*]
    \item \textbf{Experience required}: 2--4 years of technical engineering 
      experience.
    \item \textbf{Education}: Bachelor's Degree (Computer Science or 
      Engineering) required.
    \item \textbf{Support}: Reasonable accommodation for the application or 
      interview process is available for individuals with disabilities. 
      (Relocation assistance is not specified).
    \item \textbf{Travel}: Required (Yes).
  \end{enumerate}
\end{itemize}

\section*{Career Path}
T-Mobile describes career growth as a \textbf{``jungle gym of possibilities''} 
rather than a traditional corporate ladder. This role is positioned as an 
investment in career growth, with an emphasis on learning and applying 
new technologies.

\newpage
\subsection*{Core responsibilities}
\begin{itemize}
  \item \textbf{Technical Engineering Services:} Develop software solutions; 
    conduct tests and inspections; prepare reports and calculations. 
    Expected to independently develop a full software stack.
  \item \textbf{Technical Leadership:} Collaborate with technical teams and 
    apply system expertise to deliver solutions; continuously learn new 
    technologies.
  \item \textbf{Technical Writing:} Write documentation on how technology 
    works; document system design, presentations, and business requirements.
  \item \textbf{Technology Strategy:} Contribute to emerging technologies to 
    deliver business goals; interact with system engineers to define 
    requirements.
  \item \textbf{Innovation:} Present new ideas to improve existing 
    systems/processes; review current company processes to highlight 
    opportunities for refinement.
  \item U.S. work authorization (Legally authorized to work in the United 
    States) and at least 18 years of age.
  
  \item \hrulefill
  
  \item \textbf{Required Skills:}
  \begin{itemize}
    \item Communication
    \item Customer Service
    \item Analytics
    \item Technical Writing
  \end{itemize}
  
  \item \textbf{Preferred Stack (Technical Skills):}
  \begin{itemize}
    \item .net
    \item Containerization (Docker, Kubernetes)
    \item CI/CD (git, GitLab, Bitbucket, Jenkins, or similar)
    \item React (for UI)
    \item MSSQL
    \item Splunk
    \item Grafana
  \end{itemize}
  
  \item \hrulefill
  \item \textbf{Software Defined Networking (SDN):} A network architecture 
    approach that enables the network to be intelligently and centrally 
    controlled, or ``programmed,'' using software applications.
  \item \textbf{Network Functions Virtualization (NFV):} The concept of 
    replacing dedicated network appliances (like routers and firewalls) 
    with software running on standard IT infrastructure (servers, storage, 
    switches).
  \item \textbf{Microservices:} An architectural style that structures an 
    application as a collection of loosely coupled, independently 
    deployable services.
  \item \textbf{Containerization (Docker, Kubernetes):} A lightweight form of 
    virtualization used to run and manage microservices and applications 
    (Docker) and to automate their deployment, scaling, and operation 
    (Kubernetes).
  \item \textbf{CI/CD (Continuous Integration/Continuous Deployment):} A set 
    of practices (like using Git, Jenkins) that automate the software 
    build, test, and deployment pipeline.
\end{itemize}

In this role, the Software Engineer is responsible for the architecture and 
design of software components, leveraging virtualization, microservices, 
SDN, NFV, and Big Data platforms to meet customer requirements.


\end{document}