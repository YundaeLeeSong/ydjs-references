% One-page, ATS-friendly LaTeX resume + cover letter (Automation Genesis — Discrete Engineering Track)
% Replace all bracketed placeholders [like this] with your personal details.
\documentclass[11pt]{article}
\hbadness=10000                       % Suppress Warnings (Horizontal)
\vbadness=10000                       % Suppress Warnings (Vertical)
\usepackage[margin=0.7in]{geometry}
\usepackage{parskip}          % better spacing between paragraphs
\usepackage{enumitem}         % control list spacing
\usepackage[hidelinks]{hyperref}
\usepackage{helvet}

% Compact list spacing
\setlist[itemize]{leftmargin=*,itemsep=2pt,topsep=2pt}
\setlist[enumerate]{leftmargin=*,itemsep=2pt,topsep=2pt}

\begin{document}

% https://careers.salesforce.com/en/jobs/jr312082/software-engineer-mts-digital-enterprise-technology-det/



\section*{Software Engineer, MTS – Digital Enterprise Technology (DET) at Salesforce}

The \textbf{Software Engineer, Member of Technical Staff (MTS)} in the 
\textbf{Digital Enterprise Technology (DET)} division focuses on developing 
Web Services and Salesforce platform engineering. This role is responsible 
for designing, building, testing, and maintaining scalable Salesforce 
applications and web services using technologies like Apex, Lightning Web 
Components (LWC), Java, and Python. The position is a hands-on engineering 
role covering the full development lifecycle, emphasizing modern DevOps and 
Quality Engineering practices.

\section*{Software Engineering by \textbf{Salesforce}}
\textbf{Job ID:} JR312082. This role is part of the DET team, which acts as 
\textbf{``Customer Zero,''} leading by example to showcase what is possible 
with Salesforce technology at scale.

\begin{itemize}
  \item \textbf{Field:} Software Engineering / Digital Enterprise Technology
  \item \textbf{Pay range:} (Not specified in the description)
  \item \textbf{Benefits:} (Not specified in the description)
  \item \textbf{Employment type:} Full time
  \item \textbf{Locations:} Indiana - Indianapolis; Texas - Dallas
  \item Posted on \textbf{21 October 2025}
  \begin{enumerate}[label=\arabic*., leftmargin=*]
    \item \textbf{Experience required}: 3–5 years of experience in software 
      development.
    \item \textbf{Education}: (Not specified; inferred by experience level).
    \item \textbf{Support}: (Relocation assistance not specified).
    \item \textbf{Travel}: (Not specified).
  \end{enumerate}
\end{itemize}

\section*{Career Path}
This role is situated within the Digital Enterprise Technology (DET) team, 
which guides Salesforce's internal digital transformation. The position is 
described as a hands-on engineering role for a \textbf{self-starter} who 
thrives in a fast-paced environment. It involves taking ownership and 
communicating effectively across global teams, suggesting a path focused on 
deep technical expertise and cross-functional influence within the enterprise.

\newpage
\subsection*{Core responsibilities}
\begin{itemize}
  \item Design, build, test, and maintain scalable Salesforce applications 
    and web services.
  \item Implement modern DevOps and Quality Engineering practices, including 
    automated testing, code quality enforcement, and environment management.
  \item Utilize tools such as Git, Jenkins, SFDX, Salesforce CLI, and 
    Salesforce APIs.
  \item Collaborate across teams to streamline workflows, ensure security 
    best practices, and document technical solutions.
  \item Support continuous improvement in release and operations processes.
  \item Cover all phases of the development lifecycle, from design and 
    implementation through testing and production deployment.

  \item \hrulefill
  \item \textbf{Required Skills \& Experience:}
  \begin{itemize}
    \item \textbf{Salesforce:} Deep expertise in Apex, Lightning Web Components 
      (LWC), Apex triggers, Async Apex, and Salesforce Flows (with a strong 
      understanding of performance, scalability, and security best practices).
    \item \textbf{Programming:} Proficiency in Java or Python.
    \item \textbf{Process:} Solid understanding of the Software Development 
      Life Cycle (SDLC).
    \item \textbf{Cloud/Containers (2+ yrs):} Experience with cloud platforms 
      (e.g., AWS, Google Cloud) and containerization (e.g., Docker, Kubernetes).
    \item \textbf{Databases (2+ yrs):} Experience with relational and 
      non-relational databases (e.g., Postgres, Redis, Elasticsearch, MongoDB).
    \item \textbf{QA/Automation (2+ yrs):} UI automation experience using 
      Selenium, TestNG, and Jenkins.
    \item \textbf{API Testing (2+ yrs):} Experience in testing web services 
      (SOAP/REST APIs).
  \end{itemize}
  
  \item \textbf{Desired Skills:}
  \begin{itemize}
    \item \textbf{Certifications:} Salesforce Administrator and/or Salesforce 
      Platform Developer.
    \item \textbf{AI Implementation:} Experience with Agentforce for AI agent 
      deployments.
    \item \textbf{AI Development:} Familiarity with AI-assisted development 
      tools (e.g., Cursor).
    \item \textbf{UI Development:} 2+ years of experience with React.js (noted 
      as a significant plus).
    \item \textbf{Methodology:} Experience developing in an Agile environment.
    \item \textbf{Practices:} Experience with Test-Driven Development (TDD) 
      and Continuous Integration (CI).
    \item \textbf{Model:} Understanding of Software-as-a-Service (SaaS) models.
  \end{itemize}
  
  \item \hrulefill
  \item \textbf{Apex:} Salesforce's proprietary, back-end programming language, 
    which has a Java-like syntax. It is used to execute flow and transaction 
    control statements on the Salesforce platform.
  \item \textbf{Lightning Web Components (LWC):} A modern UI framework for 
    building front-end applications on the Salesforce platform using standard 
    HTML, JavaScript, and CSS.
  \item \textbf{SFDX (Salesforce DX):} A set of tools and a development 
    methodology (Salesforce Developer Experience) that facilitates 
    source-driven development, team collaboration, and automated CI/CD 
    pipelines for Salesforce.
  \item \textbf{Agentforce:} A specific implementation (likely internal to 
    Salesforce) related to the deployment of AI agents.
\end{itemize}

In this role, the engineer is responsible for hands-on, full-lifecycle 
development within the Salesforce ecosystem, utilizing both proprietary 
(Apex, LWC) and open-source (Java, Python, Kubernetes) technologies to 
drive Salesforce's internal digital strategy.

\newpage



\begin{tabular}{|l|l|l|}
\hline Salesforce Title & Common Equivalent & Notes \\
\hline Software Engineer (SWE) & Entry-Level / Junior Engineer & Usually for new grads or early-career (0-2 yrs). Look for this if you're just starting out. \\
\hline Associate MTS (AMTS) & Entry-Level SWE & "MTS" stands for Member of Technical Staff. "Associate" is the lowest level-ideal for entry roles. \\
\hline MTS (Member of Technical Staff) & Mid-Level SWE & Requires 2-5 yrs of experience; not entry level. \\
\hline Senior MTS (SMTS) & Senior SWE & Experienced engineer, usually 5-8 yrs+. \\
\hline Lead MTS (LMTS) & Staff / Tech Lead & High-level individual contributor, typically 8-10 yrs+. \\
\hline Principal MTS (PMTS) & Principal Engineer & Top-tier IC, often 10+ yrs or recognized expert. \\
\hline
\end{tabular}



\end{document}