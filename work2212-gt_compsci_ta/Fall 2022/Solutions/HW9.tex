\documentclass{article}
\usepackage[utf8]{inputenc}
\usepackage{xcolor}
\usepackage{comment}
\usepackage{setspace}
\usepackage{algpseudocode}
\usepackage{mathtools}

\usepackage{enumitem}

\usepackage{amsmath, amssymb, amsthm}


\title{CS 2050 Fall 2022 Homework 9}
\author{Due: November 4}
\date{Released: November 11}

\newcommand{\pt}[1]{\textcolor{blue}{(#1 points)}}

\newcommand{\pte}[1]{\textcolor{blue}{(#1 points each)}}

\newenvironment{solution}
{
\par
\color{blue}
\textbf{Solution:}
}
{
\par
}

\newenvironment{rubric}
{
\par
\begin{spacing}{.6}
\begin{itshape}
\color{red}

}
{
\end{itshape}
\end{spacing}
\par
}

\begin{document}

\maketitle

\begin{enumerate}
    \item[i.] This assignment is due on \textbf{11:59 PM EST, Friday, November 11, 2022}.  On-time submissions receive 2.5 points of extra credit. You may turn it in one day late for a 10 point penalty or two days late for a 25 point penalty. Assignments more than two days late will NOT be accepted.  We will prioritize on-time submissions when grading before an exam.
    \item[ii.] You will submit your assignment on \textbf{Gradescope}. Shorter answers may be entered directly into response fields, however longer answer must be recorded on a typeset (e.g. using \LaTeX) or \emph{neatly} written PDF.
    \item[iii.] Ensure that all questions are correctly assigned on Gradescope. Questions that take up multiple pages should have all pages assigned to that question. Incorrect page assignments can lead to point deductions.
    \item[iv.] You may collaborate with other students, but any written work should be your own. Write the names of the students you work with on the top of your assignment.
    \item[v.] Always justify your work, even if the problem doesn't specify it. It can help the TA's to give you partial credit.
\end{enumerate}

Author(s): David Teng, Richard Zhao, Sarthak Mohanty, Rohan Bodla
\newpage
\begin{enumerate}

\item Show that a three-dimensional $2^n$ × $2^n$ × $2^n$ checkerboard with one 1 × 1 × 1 cube missing can be completely filled by 2 × 2 × 2 cubes with one 1 × 1 × 1 cube removed.
\begin{solution}
Proof: Let $P(n)$ be that a $2^n$ x $2^n$ x $2^n$ checkerboard with one 1 x 1 x 1 cube missing can be filled by 2 x 2 x 2 cubes that have one 1 x 1 x 1 cube removed. I proceed by math induction to show that $P(n)$ holds true for all integers $n \geq 1$.

Basis Step: $P(1)$ holds true because a $2^1$ x $2^1$ x $2^1$ checkerboard with one 1 x 1 x 1 cube missing is congruent to a 2 x 2 x 2 cube with one 1 x 1 x 1 cube removed. There are 8 possible positions for the missing corner, and the basis step holds true because we can rotate our piece accordingly.

Inductive Hypothesis: Assume $P(k)$: a $2^k$ x $2^k$ x $2^k$ checkerboard with one 1 x 1 x 1 cube missing can be filled by 2 x 2 x 2 cubes with one 1 x 1 x 1 cube removed.\\
Inductive Step: I Proceed to show that $P(k) \rightarrow P(k + 1)$ is true whenever $P(k)$ is true.

Let us consider that a $2^{k+1}$ x $2^{k+1}$ x $2^{k+1}$ cube can be divided into eight $2^k$ x $2^k$ x $2^k$ cubes by splitting every side in half.\\
1) One of the eight $2^k$ x $2^k$ x $2^k$ cubes has a square missing, and by the Inductive Hypothesis it can be tiled by $2$ x $2$ x $2$ cubes with one 1 x 1 x 1 cube removed.\\
2) For the remaining seven $2^k$ x $2^k$ x $2^k$ cube, they can be built with seven $2^k$ x $2^k$ x $2^k$ checkerboards such that the missing cube is in the corner. By the Inductive Hypothesis, each of these boards can be tiled with 2 x 2 x 2 cubes with a 1 x 1 x 1 cube missing. The seven large checkerboards can be arranged together such that the seven missing 1 x 1 x 1 cubes are adjacent, and those missing cubes can be filled by a 2 x 2 x 2 cube that is missing a 1 x 1 x 1 cube.\\
We have shown that we can make seven complete $2^k$ x $2^k$ x $2^k$ cubes and an eighth $2^k$ x $2^k$ x $2^k$ checkerboard with one 1 x 1 x 1 cube missing, hence the entire $2^{k+1}$ x $2^{k+1}$ x $2^{k+1}$ checkerboard with one cube missing can be covered with 2 x 2 x 2 cubes with a 1 x 1 x 1 cube missing. $P(k+1)$ is true whenever $P(k)$ is true. This completes the inductive step.

By math induction $P(n)$ is true for all $n \geq 1$.

\end{solution}

%\item Prove that $\sqrt[x]{2}$ is irrational, for all integers $x > 2$. You can use the fact that $a^n + b^n = c^n$ is always false for any 3 integers a,b and c when $n > 2$.

\item Given sets $A_1, A_2, A_3, ..., A_n$ that are a part of a universal set $U$, prove using strong induction that $\overline{A_1 \cup A_2 \cup A_3 \cup ... \cup A_n} = \overline{A_1} \cap \overline{A_2} \cap \overline{A_3} \cap ... \cap \overline{A_n}$.
\begin{solution}\\
\textbf{NOTE: Math induction can be used instead of strong induction}

Proof: Let $P(n)$ be that $\overline{A_1 \cup A_2 \cup A_3 \cup ... \cup A_n} = \overline{A_1} \cap \overline{A_2} \cap \overline{A_3} \cap ... \cap \overline{A_n}$. I proceed by strong induction to show that $P(n)$ holds true for all integers $n \geq 1$.

Basis Step: $P(1)$ holds true because $\overline{A_1} = \overline{A_1}$.

Inductive Hypothesis: Assume $P(j): \overline{A_1 \cup A_2 \cup A_3 \cup ... \cup A_j} = \overline{A_1} \cap \overline{A_2} \cap \overline{A_3} \cap ... \cap \overline{A_j}$ for any integer $j$ where $1 \leq j \leq k$, where $k$ is a fixed arbitrary integer such that $k \geq 1$.\\
Inductive Step: I proceed to show that $P(k + 1)$ is true whenever the inductive hypothesis is true.
\begin{center}
\begin{tabular}{l|l}
    Statement & Reasoning\\ \hline
    1) $\overline{A_1 \cup A_2 \cup A_3 \cup ... \cup A_k} =  \overline{A_1} \cap \overline{A_2} \cap \overline{A_3}     \cap ... \cap \overline{A_k}$ & Inductive Hypothesis\\
    2) Let $B = A_1 \cup A_2 \cup A_3 \cup ... \cup A_k$, $B$ is a set & Set properties\\
    3) $\overline{B} = \overline{A_1} \cap \overline{A_2} \cap \overline{A_3} \cap ... \cap \overline{A_k}$ & Substitute (2) into (1)\\
    4) $\overline{B} \cap \overline{A_{k+1}} = \overline{A_1} \cap \overline{A_2} \cap \overline{A_3} \cap ... \cap \overline{A_k} \cap \overline{A_{k+1}}$ & Intersect (3) with $\overline{A_{k+1}}$\\
    5) $\overline{B \cup A_{k+1}} = \overline{A_1} \cap \overline{A_2} \cap \overline{A_3} \cap ... \cap \overline{A_k} \cap \overline{A_{k+1}}$ & De Morgan's Law on LHS\\
    6) $\overline{A_1 \cup A_2 \cup ... \cup A_k \cup A_{k+1}} =  \overline{A_1} \cap \overline{A_2} \cap ... \cap \overline{A_k} \cap \overline{A_{k+1}}$ & Substitute (2) into (5)
\end{tabular}
\end{center}

Thus, we see that $P(k + 1)$ is true under the assumption that the inductive hypothesis is true. This completes the inductive step.

By strong induction $P(n)$ holds true for all integers $n \geq 1$. $\square$
\end{solution}

% perhaps add this to the next hw.
% \item Use strong induction to show that the square root of 28 is irrational. You must use strong induction to receive credit on this problem.

\item Find the smallest integer value $x$ such that you can form postage of $x$ cents and greater using only 6 cent and 11 cent stamps. Use strong induction to prove that your identified value of x is a valid choice.

\begin{solution} \\
The minimum value of $x$ such that you can form postage of $x$ cents and greater using only 6-cent and 11-cent stamps: 50 \\\\
Proof: Let $P(n)$ be that you can form postage of 50 cents and greater using only 6-cent and 11-cent stamps. I proceed using strong induction to show that $P(n)$ is true for all $n >= 50$. \\ \\
Basis Step: \\
$P(50)$ holds true because $50 = 11(4) + 6(1)$ \\
$P(51)$ holds true because $51 = 11(3) + 6(3)$ \\ 
$P(52)$ holds true because $52 = 11(2) + 6(5)$ \\
$P(53)$ holds true because $53 = 11(1) + 6(7)$ \\
$P(54)$ holds true because $54 = 11(0) + 6(9)$ \\
$P(55)$ holds true because $55 = 11(5) + 6(0)$ \\ \\
The base case holds true for values between 50 and 55. This concludes the base case. \\

Inductive Hypothesis: Assume P(j): Postage of $j$ cents can be formed using only 6-cent and 11-cent stamps is true for any integer $j$ where $50 <= j <= k$, where $k$ is a fixed arbitrary integer such that $k >= 55$ \\  

Inductive Step: I show $P(k+1)$ is true when the inductive hypothesis is true. Consider $(k+1)$ cent postage stamps. Subtract 6 cents. We now have $(k-5)$ cents. Note that $50 <= (k-5) <= k$ since $k >= 55$. By the inductive hypothesis $(k-5)$ cents can be formed using only 6-cent and 11-cent stamps. I use an additional 6-cent stamp to form $(k+1)$ cents. Thus we can see that $(k+1)$ cents postage can be formed using only 6-cent and 11-cent stamps. Hence, $P(k+1)$ is true whenever the inductive hypothesis is true. This completes the inductive step. \\ 

Conclusion: By strong induction, $P(n)$ is true for all integers $n>=50$.
\end{solution}

\item Suppose you have the sequence which is recursively defined as follows:\\
\\$a_0 = 1$
\\$a_1 = 4$
\\$a_2 = 16$
\\$a_t = a_{t-1}+ a_{t-2} + a_{t-3}$, where $t \in \mathbb{Z}^{\geq 3}$\\
\\Using strong induction, prove that $a_n \leq 4^n$ for all $n \geq 0$.
\begin{solution}\\
Proof: Let $P(n)$ be that $a_n \leq 4^n$. I proceed by strong induction to show that $P(n)$ holds true for all integers $n \geq 0$.

Basis Step: $P(0)$ holds true because $a_0 = 1 \leq 4^0$, $P(1)$ holds true because $a_1 = 4 \leq 4^1$, $P(2)$ holds true because $a_2 = 16 \leq 4^2$.

Inductive Hypothesis: Assume $P(j): a_j \leq 4^j$ for any integer $j$ where $0 \leq j \leq k$, where $k$ is a fixed arbitrary integer such that $k \geq 2$.\\
Inductive Step: I proceed to show that $P(k + 1)$ is true whenever the inductive hypothesis is true.
\begin{center}
\begin{tabular}{l|l}
    Statement & Reasoning\\ \hline
    1) $a_k \leq 4^k$ & Inductive Hypothesis\\
    2) $a_{k-1} \leq 4^{k-1}$ & Inductive Hypothesis\\
    3) $4^{k-1} \leq 4^k$ & $k \geq 2$ and $k > k - 1$\\
    4) $a_{k-1} \leq 4^k$ & Transitivity on (2) and (3)\\
    5) $a_{k-2} \leq 4^{k-2}$ & Inductive Hypothesis\\
    6) $4^{k-2} \leq 4^k$ & $k \geq 2$ and $k > k - 2$\\
    7) $a_{k-2} \leq 4^k$ & Transitivity on (5) and (6)\\
    10) $a_k + a_{k-1} + a_{k-2} \leq 4^k + 4^k + 4^k$ & Add (1), (4), and (7)\\
    11) $a_k + a_{k-1} + a_{k-2} \leq 3 \cdot 4^k$ & Simplify RHS of (10)\\
    12) $a_{k+1} = a_k + a_{k-1} + a_{k-2}$ & Given\\
    13) $a_{k+1} \leq 3 \cdot 4^k$ & Substitute (12) into (11)\\
    14) $3 \leq 4$ & Self-evident\\
    15) $3 \cdot 4^k \leq 4 \cdot 4^k$ & Multiply (14) by $4^k$\\
    16) $a_{k+1} \leq 4 \cdot 4^k$ & Transitivity on (13) and (15)\\
    17) $a_{k+1} \leq 4^{k+1}$ & Exponent property
\end{tabular}
\end{center}

Thus, we see that $P(k + 1): a_{k+1} \leq 4^{k+1}$ is true under the assumption that the inductive hypothesis is true. This completes the inductive step.

By strong induction $P(n)$ holds true for all integers $n \geq 0$. $\square$
\end{solution}

\end{enumerate}
\end{document}

